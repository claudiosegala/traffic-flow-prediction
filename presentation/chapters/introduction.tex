\slide{Motivação}{
    \begin{itemize}
        \item Alto volume da frota de veículos ;
        \item Congestionamentos reduzem a qualidade de vida e aumentam as emissões de gás carbono na atmosfera;
        \item Trabalhos da área tendem a focar em vias livres, que não consideram obstrução nas vias ;
        \item Este trabalho propõe cenários que se assemelham mais a vias urbanas (com cruzamentos).
    \end{itemize}
}

\slide{Objetivo}{
    Realizar a predição de fluxo de veículos em vias urbanas a curto prazo (15, 30, 45, 60 minutos no futuro). Para realizar essa tarefa foram escolhidos os modelos \textit{Support Vector Machine}, \textit{Random Forest}, \textit{Long Short-Term Memory} e \textit{Gated Recurrent Unit}.

    \begin{itemize}
        \item Implementar a análise e transformação dos dados adquiridos;
        \item Implementar os modelos de aprendizagem de máquina escolhidos;
        \item Realizar uma busca pelos melhores valores de parâmetros e hiper-parâmetros;
        \item Treinar os modelos com os dados processados;
        \item Automatizar a comparação dos modelos;
    \end{itemize}
}
