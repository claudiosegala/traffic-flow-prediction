\slide{Limitações}{
    \begin{itemize}
        \item Fator Espacial
        \item Fator Temporal
        \item Fator Aleatório
        \item Conjunto de Dados
    \end{itemize}
}

\slide{Conjunto de Dados}{
    \begin{itemize}
        \item Doutorando Fábio e DETRAN
        \item Avenida Hélio Prates
        \item Maio a Junho de 2016
        \item Resgistro de Veículos
    \end{itemize}
}

\slide{Conjunto de Dados}{
    \begin{table}[h]
        \begin{tabular}{ccccccc}
        \toprule
        \multicolumn{1}{l}{\textbf{Id Equi.}} & \multicolumn{1}{l}{\textbf{A/M/D}} & \multicolumn{1}{l}{\textbf{Hora}} & \multicolumn{1}{l}{\textbf{Faixa}} & \multicolumn{1}{l}{\textbf{km/h}} & \multicolumn{1}{l}{\textbf{km/h Max}} & \multicolumn{1}{l}{\textbf{Tam.} \\ 
        \midrule
        RSI128 & 2016/5/1 & 00:00:09 & 1 & 20 & 60 & 0 \\
        RSI131 & 2016/5/1 & 00:00:09 & 2 & 45 & 60 & 1.1 \\
        RSI132 & 2016/5/1 & 00:00:09 & 1 & 40 & 60 & 0 \\
        RSI131 & 2016/5/1 & 00:00:10 & 1 & 35 & 60 & 0.5 \\ 
        \bottomrule
        \end{tabular}
        \label{table:data}
        \caption{Exemplo dos dados recebidos coletados pelo \acrshort{DETRAN}}
    \end{table}
}

\slide{Pré-Processamento dos Dados}{
    \begin{itemize}
        \item Um sensor
        \item Remoção de Atributos
        \item One-Hot Encoding
    \end{itemize}
}

\slide{Compactação dos Dados}{
    \begin{itemize}
        \item Fluxo
        \item Densidade
        \item Velocidade Média
    \end{itemize}
}

\slide{Modelos Clássicos Escolhidos}{
    \begin{itemize}
        \item Random Forest (RF)
        \item Support Vector Machine (SVM)
    \end{itemize}
    \footnote{ARIMA não fora escolhido por questões de limitações de hardware}
}

\slide{Modelos de Aprendizagem Profunda Escolhidos}{
    \begin{itemize}
        \item Recurrent Neural Network (RNN)
        \item Long Short-Term Memory (LSTM)
        \item Gated Recurrent Unit (GRU)
    \end{itemize}
}

\slide{Treino}{
    \begin{itemize}
        \item Memória do Passado
        \item Univariado
        \item Multivariado
        \item Janela Rolante de Treino
    \end{itemize}
}

\slide{Escolha dos Parâmetros e Hiper-parâmetros}{
    Parâmetros Gerais:
    \begin{itemize}
        \item Memória do Passado \textbf{(1h a 8h, 1h)}
        \item Janela de Treino \textbf{(0.30 a 0.80, 0.05)}
    \end{itemize}
}

\slide{Escolha dos Parâmetros e Hiper-parâmetros}{
    Específicos de Aprendizagem Profunda:
    \begin{itemize}
        \item Quantidade de Neurônios \textbf{(40 a 200, 10)}
        \item Funções de Ativação \textbf{(Sigmoid, ReLu e Softmax)}
        \item Otimizador do Modelo \textbf{(Adam, Adamax)} % https://keras.io/optimizers/
    \end{itemize}
}

\slide{Escolha dos Parâmetros e Hiper-parâmetros}{
    Específicos de Random Forest:
    \begin{itemize}
        \item Números de Estimadores \textbf{(50 a 400, 50)}
        %\item Altura Máxima \textbf{(TODO)}
        \item Bootstrapping \textbf{(True ou False)}
    \end{itemize}
}

\slide{Escolha dos Parâmetros e Hiper-parâmetros}{
    Específicos de SVM:
    \begin{itemize}
        \item TODO!
    \end{itemize}
}

\slide{Métricas de Avaliação}{
    \begin{itemize}
        \item Raiz do Erro Médio Quadrádico
        \item Raiz do Erro Médio Quadrádico Normalizado
        \item Erro Máximo Absoluto
        \item Precisão de Tendência
    \end{itemize}
}