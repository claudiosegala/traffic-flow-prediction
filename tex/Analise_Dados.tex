% TODO: Devemos explicar como o fiscalizador eletrônico funciona? (https://industriahoje.com.br/como-funcionam-os-radares-de-velocidade)

Esse capítulo tem como objetivo mostrar uma análise dos tipos de dados que estamos lidando. Mostrando os desafios e obstáculos que esse dados possuem. Além disso, iremos especificar melhor os dados que usamos para esse trabalho.

\section{Características do Tráfego}

É importante notar que o tráfego é afetado por fatores temporais, espaciais e aleatórios. Os fatores temporais são, por exemplo, o horário de entrada e saída dos trabalhadores. Os fatores espaciais são, por exemplo, como a malha rodoviário se distribui em uma localização ou a quantidade de empresas em uma área. Já fatores aleatórios são, por exemplo, acidentes e a meteorologia do dia. Sendo assim, modelos de previsões podem ter desempenhos diferentes dependendo de onde e quando os dados forem coletados. Isso, sem levar em conta a impossibilidade de se prever os fatores aleatórios.

Assim, é comum na literatura optar por usar apenas dados com fatores espaciais e temporais. Assim, fazendo uma suposição de que exista uma tendência no fluxo que pode ser previsto e que embora as variáveis aleatórias possam afetar o fluxo, elas não afetam o bastante para mudar a tendência no longo prazo. Seguindo a literatura, esse artigo utilizará variáveis temporais e espaciais.

\section{Aquisição}

Os dados\footnote{http://bit.ly/processed-data-2l5MaAG} provêm de dois cruzamentos na avenida Hélio Prates que cruza a Taguatinga e Ceilândia, duas cidades satélites do Distrito Federal (DF). Estes dados foram coletados e fornecidos pelo Departamento de Trânsito (DETRAN\footnote{http://www.detran.df.gov.br/}) do DF. Sendo a coleta feita por fiscalizadores eletrônicos localizados nos cruzamentos.

% TODO: colocar todos os fiscalizadores eletronicos
% TODO: consertar o intervalo que recebemos
Nesses dados estão inclusos registros de todos os veículos que passaram pelo local nos meses de maio a junho de 2016 em forma de uma série temporal. Para cada registro se tem uma identificação do fiscalizador eletrônico, data, hora, faixa de via, velocidade, limite de velocidade da via e tamanho do veículo, assim como mostrado na tabela {table:data}.

\begin{table}[h]
    \begin{tabular}{ccccccc}
    \toprule
    \multicolumn{1}{l}{\textbf{Id Equipamento}} & \multicolumn{1}{l}{\textbf{Data}} & \multicolumn{1}{l}{\textbf{Hora}} & \multicolumn{1}{l}{\textbf{Faixa}} & \multicolumn{1}{l}{\textbf{km/h}} & \multicolumn{1}{l}{\textbf{km/h Max}} & \multicolumn{1}{l}{\textbf{Tamanho}} \\ 
    \midrule
    RSI128 & 2016/05/01 & 00:00:09 & 1 & 20 & 60 & 0 \\
    RSI131 & 2016/05/01 & 00:00:09 & 2 & 45 & 60 & 1.1 \\
    RSI132 & 2016/05/01 & 00:00:09 & 1 & 40 & 60 & 0 \\
    RSI131 & 2016/05/01 & 00:00:10 & 1 & 35 & 60 & 0.5 \\ 
    \bottomrule
    \end{tabular}
    \label{table:data}
    \caption{Exemplo dos dados recebidos coletados pelo \acrshort{DETRAN}}
\end{table}

Os dados foram obtidos do equipamento de fiscalização eletrônica, os quais estão instalados em vias semafóricas. Por este motivo, é possível identificar momentos curtos onde não há fluxo de carros. Além disso, alguns dos equipamentos sofreram manutenção, resultando em alguns momentos mais longos sem registro de tráfego.

\begin{figure}[t]
    \centering
    \includegraphics[scale=0.5]{street.png}
    \label{figure:eixo}
    \caption{Visão espacial do eixo monumental em Brasília}
\end{figure}


\section{Observações}

% TODO: colocar referencia do tráfego piorando durante a Copa do Mundo
Vale notar que os dados adquiridos foram de uma época com um tráfego mais intenso nas cidades do Brasil. Visto que a Copa do Mundo foi durante o intervalo adquirido. 

Também vale notar que os fiscalizadores eletrônicos utilizados ficam próximos de semáforos, ou seja, existe momentos que o fluxo diminui bastante.
