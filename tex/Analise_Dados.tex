Esse capítulo tem como objetivo mostrar uma análise dos tipos de dados que estamos lidando. Mostrando os desafios e obstáculos que esse dados possuem. Além disso, iremos especificar melhor os dados que usamos para esse trabalho.

\section{Dados de Tráfego}

Para se coletar dados de tráfego é comumente utilizado câmeras e sensores para registrar a passagem de carros numa via. Esse equipamento é utilizado bastante para fiscalizar se os carros estão obedecendo a velocidade máxima de via, sendo assim chamados de Fiscalizadores Eletrônicos.

\section{Tráfego}

É importante notar que o tráfego é afetado por fatores temporais, espaciais e aleatórios. Os fatores temporais são, por exemplo, o horário de entrada e saída dos trabalhadores. Os fatores espaciais são, por exemplo, como a grade rodoviário se distribui em uma localização ou a quantidade de empresas em uma área. Já fatores aleatórios são, por exemplo, acidentes e a meteorologia do dia. Sendo assim, modelos de previsões podem ter desempenhos diferentes dependendo de onde e quando os dados forem coletados. Isso, sem levar em conta a impossibilidade de se prever os fatores aleatórios.

Assim, é comum ver que os 



Como as variáveis aleatórias são muito difíceis de prever, a literatura opta por usar apenas dados com variáveis espaciais e temporais. Assim, fazendo uma suposição de que exista uma tendência no fluxo que pode ser prevista e que embora as variáveis aleatórias afetem o fluxo, elas não afetam o bastante para mudar a tendência do fluxo no longo prazo. Seguindo a literatura, esse artigo utilizará variáveis temporais e espaciais.

O tráfego pode ser visto de várias formas diferentes. Pode ser visto pela quantidade de veículos que passaram um ponto em um intervalo de tempo, o fluxo. Também pode ser visto como a média de velocidade dos veículos em uma certo espaço, a densidade. Ou como uma média de velocidade em um intervalo de tempo, a velocidade média.

% TODO: conferir isso
Para o nosso trabalho escolhemos utilizar o fluxo, pois seus valores estão inclusos no conjunto dos números inteiros positivos. Já os outros jeitos de ver o tráfego tem seus valores inclusos no conjunto dos números naturais. Como o domínio do fluxo é menor que os outros, os modelos tem uma maior facilidade de fazer a previsão.

\section{Dados de Tráfego}

\section{Aquisição}

Os dados\footnote{http://bit.ly/processed-data-2l5MaAG} provêm de dois cruzamentos na avenida Hélio Prates que cruza a Taguatinga e Ceilândia, duas cidades satélites do Distrito Federal (DF). Estes dados foram coletados e fornecidos pelo Departamento de Trânsito (DETRAN\footnote{http://www.detran.df.gov.br/}) do DF. Sendo a coleta feita por fiscalizadores eletrônicos localizados nos cruzamentos.

% FIX: nos recebemos um intervalo maior dos dados, apenas pegamos e selecionamos o melhor intervalo
Nesses dados estão inclusos registros de todos os veículos que passaram pelo local nos meses de maio a junho de 2016 em forma de uma série temporal. Para cada registro se tem uma identificação do fiscalizador eletrônico, data, hora, faixa de via, velocidade e limite de velocidade da via, assim como mostrado na tabela \ref{table:sampleDETRAN}.

\begin{table}[h]
    \caption{Amostra dos dados recebidos coletados pelo DETRAN}
    \label{table:sampleDETRAN}
    \begin{center}
    \begin{tabular}{ccccccc}
    \hline
    \multicolumn{1}{l}{\textbf{Equip.}} & \multicolumn{1}{l}{\textbf{Data}} & \multicolumn{1}{l}{\textbf{Hora}} & \multicolumn{1}{l}{\textbf{Faixa}} & \multicolumn{1}{l}{\textbf{Vel.}} & \multicolumn{1}{l}{\textbf{Lim. Vel.}}\\ 
    \hline
    RSI128 & 2016/05/01 & 00:00:09 & 1 & 20 & 60 \\
    RSI131 & 2016/05/01 & 00:00:09 & 2 & 45 & 60 \\
    RSI132 & 2016/05/01 & 00:00:09 & 1 & 40 & 60 \\
    RSI131 & 2016/05/01 & 00:00:10 & 1 & 35 & 60 \\ 
    \hline
    \end{tabular}
    \end{center}
\end{table}