% TODO: somente descrever 4 trabalhos, como o Geraldo mostrou
% TODO: citar mais coisas em que GAN é utilizado
% TODO: corrigir artigo de origem do RCGAN

% Falar sobre a SAE e do principal trabalho relacionado
Esse trabalho foi inspirado fortemente no trabalho de \textit{traffic flow prediction} usando \acrshort{SAE} \cite{lv_6894591}.



% Falar sobre GAN
\acrshort{GAN} foi introduzida em 2014 no artigo \cite{NIPS2014_5423} e é possível encontrar pesquisas na área de geração de imagens a partir de textos \cite{reed2016generative}.

% Falar sobre o problema da GAN para os nossos dados
Porém essas pesquisas utilizam de dados independentes e a \acrshort{GAN} tentava descobrir a distribuição desses dados. No nosso caso, os dados estão associados a um tempo, o que dá a eles uma certa dependência entre si (esses dados ocorrem antes desses outros). Mas mesmo assim, existem trabalhos que também utilizam \textit{time series} e alcançaram resultados promissores na área de previsão de ações \cite{zhou_2018} \cite{banushev_2019} e geração de dados temporais médicos \cite{esteban2017real}.

Nesses trabalhos são usados variações da \acrshort{GAN} para que ela seja capaz de trabalhar com \textit{time series}. Sendo uma das variações a \acrfull{InfoGAN} \cite{chen2016infogan}, \acrfull{ACGAN} \cite{odena2017conditional} e \acrfull{RCGAN} \cite{esteban2017real}.

% Falar da otimização (MH-GAN)
Além disso, é possível melhorar a performance do gerador de uma \acrshort{GAN} utilizando o discriminador que normalmente é jogado fora, segundo \cite{turner2018metropolis}. \acrfull{MH-GAN} teve resultados promissores.

\section{Fundamentação Teórica}

\subsection{RNN}

Redes neurais recorrentes são um tipo de rede neural aonde a saída depende não somente da entrada, mas também dos estados anteriores da rede, o que acaba por agir como um tipo de memória. Neste trabalho, utilizamos uma variação desse tipo de rede neural chamada \acrfull{LSTM}.


\subsection{LSTM}

O LSTM é primeiramente citado em \cite{Felix_1999} aonde o autor define pela primeira vez o seu conceito e funcionamento. Com o passar dos anos e o aumento do número dos carros nas rodovias, o LSTM foi rapidamente inserido no contexo de traffic prediction flow. Existem diversos artigos e trabalhos a respeito do seu uso dentro deste assunto. Um dos mais relevantes deles é o \cite{Xiaolei_2015}, aonde o autor faz uma comparação do desempenho do LSTM com outros métodos de predição. 





%    - Entrada: Camada que recebe os dados 
%    - \arcshort{LSTM}: (camada aonde ocorre a recursão dos dados e as células de memória são atualizadas, %decidindo o que continua na rede e o que é esquecido)
%    - Saída:  

\subsection{GAN}

Para fazer a previsão será utilizada \acrfull{GAN}. \acrshort{GAN} é um método que utiliza dois modelos para serem o gerador e o discriminador. O trabalho do gerador é tentar reproduzir algo. O trabalho do discriminador é avaliar o que o gerador fez e decidir se é bom o suficiente. \acrshort{GAN} põe dois modelos para treinarem juntos como rivais. Neste trabalho escolhemos como modelo gerador a \acrshort{LSTM} e como modelo discriminador o --indefinido--

\subsection{MH-GAN}

No caso, o artigo de previsão de ações utilizou uma variação da \acrshort{GAN} desenvolvida pela \textit{Uber} e tiveram resultados que superaram o estado da arte. A empresa estadunidense introduziu a \acrfull{MH-GAN} em \cite{turner2018metropolis} que tem como objetivo fazer com que a \acrshort{GAN} sirva para treinar o gerador. 