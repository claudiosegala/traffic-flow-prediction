% Motivação Social
% TODO: adicionar mais problemas com o crescimento dos carros
% TODO: adicionar os fatores externos
O volume de carros tem aumentado rapidamente ao longo dos últimos anos e as vias públicas não tem conseguido acompanhar esse crescimento. Com esse crescimento vêm problemas como congestionamentos que iram se tornar mais frequentes. Sendo assim, o controle de congestionamento é um assunto que merece cada vez mais atenção nos grandes centros urbanos. Visto que com engarrafamentos menos severos, vem também uma diminuição do tempo gasto no trânsito, assim como uma redução da gasolina utilizada e, consequentemente, a quantidade de Co2 emitido na atmosfera (reduzindo o impacto ambiental).

Um dos meios de amenizar o congestionamento é através da predição de tráfego nas vias. Caso as predições sejam disponibilizados para cidadãos e empresas de mobilidade urbana por meio de alguma aplicação, como o \textit{Uber}, ou \textit{Waze}, os motoristas podem tomar decisões de rotas mais inteligentes. Reduzindo assim os congestionamento, visto uma melhor distribuição dos carros na malha viária.

% Motivação Científica
% TODO: adicionar referências para artigos que utilizam metodologias heurísticas
Visto os benefícios, existem diversos trabalhos nessa área (predição de tráfego). As metodologias usadas giram em torno de predições baseadas em aprendizagem profunda \cite{Xiaolei_2015} \cite{Zainab_2018} \cite{he2013improving} \cite{wang_2018} \cite{lv_6894591} e predições heurísticas. Os heurísticos levam em conta que os dados dos veículos podem ser vistos como séries temporais para fazer as previsões e normalmente são variações do \acrfull{ARIMA}. Já o estado da arte está entre os métodos que utilizam aprendizagem profunda usam \acrfull{SLSTM}) e \acrfull{SAE}. Porém os resultados, embora muito bons, ainda tem espaço para melhora.

% Justificativa
% TODO: descobrir características positivas da GAN e negativas da LSTM para fazer uma melhor justificativa.
Uma nova metodologia de aprendizagem profunda vêm trazendo resultados promissores em diversas áreas: \acrfull{GAN}. Até o momento da escrita de artigo não foram encontrados pesquisas na área de previsão de tráfego. Porém existem trabalhos em áreas que utilizam dados semelhantes aos usados em previsão de tráfego que trouxeram resultados promissores. 

\section{Metodologia}
Todas essas técnicas de predição de tráfego utilizam de dados do passado para extrair um padrão. E para se extrair esses dados é preciso utilizar uma infraestrutura local nas vias, como uma fiscalização eletrônica, ou uma infraestrutura global, como atualização da sua posição atual pelo celular. No nosso trabalho utilizaremos de dados da infraestrutura local, mais especificamente a fiscalização eletrônica local coletada diretamente do \acrfull{DETRAN} do \acrfull{DF}. Os dados são de maio de 2018 até junho de 2018 e eles possuem os seguintes parâmetros:

% TODO: transformar isso numa tabela
\begin{enumerate}
    \item Velocidade Limite
    \item Velocidade
    \item Tempo
    \item Via
    \item Tamanho do Carro
\end{enumerate}

% TODO: [LATER] adicionar uma gráfico que mostra como estão os dados (fluxo por tempo, por exemplo).

O tráfego é afetado por variáveis temporais (hora do dia influencia o fluxo), espaciais (vias de alta movimentação) e aleatórias (acidentes, chuvas). Geralmente, só são usados os dados temporais e espaciais, visto que é muito difícil coletar e tentar prever os dados aleatórios, assim assumindo que:
\begin{itemize}
    \item \textbf{A história vai se repetir}, isto é, existem tendências nos dados.
    \item \textbf{Eventos aleatórios não afetam a tendência}, isto é, é normal que existam eventos randômicos que alterem o fluxo, porém estes não vão mudar o comportamento de forma global.
\end{itemize}

Porém existem pesquisas que tentam aprimorar o estado da arte tentando coletar e utilizar das variáveis aleatórias provindas de redes sociais como o Twitter \cite{he2013improving} ou utilizam de dados meteorológicos \cite{wang_2018}. Mas no geral, essas variáveis são ignoradas.

Para tentar superar o estado da arte vamos utilizar de uma variação da \acrshort{GAN} com \acrshort{LSTM} como gerador e um \acrshort{CNN} como discriminador para tentar gerar a série temporal que seriam a predição do tráfego.

% TODO: [LATER] adicionar uma subseção de modelagem do problema da parte de metodologia (lembrar Geraldo)

\section{Hipótese}
% TALVEZ: criar mais hipóteses que sejam mais fáceis de provar
% TALVEZ: comparar as duas quando se adiciona chuva
% TALVEZ: tentar prever até um dia na frente
% TALVEZ: avaliar a importância dos atributos que eu tenho
% TALVEZ: pegar mais fontes de dados e ver como reage
% TALVEZ: avaliar melhores janelas de tempo para o algoritmo
Queremos então verificar se utilizarmos variações de \acrshort{GAN}, seria uma abordagem melhor que é o estado da arte atualmente (\acrshort{LSTM}). Além disso, pretendemos verificar a performance da arquitetura tanto para predição de curto período (15, 30, 45 e 60 minutos) quanto médio período (2, 4 e 6 horas).

\section{Objetivos}

% Gerais
Este artigo tem como objetivo introduzir o conceito de \acrshort{GAN} aos estudos de previsão de tráfego e mostrar seu desempenho em relação aos métodos mais utilizados atualmente. Além da introdução de uma nova metodologia na área, também deve ser possível avaliar o desempenho e a qualidade da predição para diferentes janelas de tempo, mostrando assim, qual o melhor intervalo. Paralelamente, também serão realizados testes para verificar se há a possibilidade de se utilizar um modelo baseado em uma infraestrutura de captação de dados local com dados globais. 

% Específicos
Levando em conta que o estado da arte de predição de tráfego utiliza \acrshort{LSTM}, é natural afirmar que um dos principais pontos desse trabalho é demonstrar a eficácia dessa nova metodologia (\acrshort{GAN} com \acrshort{LSTM} e \acrshort{CNN}) se comparada ao \acrshort{LSTM} puro, mais especificamente se a \acrshort{GAN} tem uma capacidade de previsão superior.

\section{Resultados Esperados}
Para avaliar as abordagens, serão usados três funções de comparação, sendo elas \acrfull{MAE}, \acrfull{RMSE}, \acrfull{MRE}. Essas são três das mais famosas funções para se medir a acurácia de dados provindos de variáveis contínuas. 

% TODO: this looks a lot like another paper, this is a problem?
% TODO: adicionar métricas de acurácia (MAE, MRE e RMSE não são métricas de acurácia)

\begin{equation}
\acrshort{MAE} = \frac{1}{n} \times \sum_{i=1}^{n} \quad \abs{result_i - expected_i}
\end{equation}

\begin{equation}
\acrshort{MRE} = \frac{1}{n} \times \sum_{i=1}^{n} \quad \frac{\abs{result_i - expected_i}}{result_i}
\end{equation}

\begin{equation}
\acrshort{RMSE} = \sqrt{ \frac{1}{n} \times \sum_{i=1}^{n} \quad (result_i - expected_i) ^ 2}
\end{equation}

Elas serão utilizadas nos resultados obtidos pelas arquiteturas para que possamos compará-las. Espera-se que a nossa abordagem se saia significativamente melhor que a abordagem que usa um \acrshort{LSTM} simples em todas essas categorias.
    
\section{Limitações}
% Quantidade de dados, Qualidade e incertezas nos dados
Para se fazer a previsão serão desconsiderados certos cenários que são muito difíceis de capturar os dados e relaciona-los. Entre eles estão:
\begin{itemize}
    \item Perda de dados pela manutenção da infraestrutura analisada (falhas no equipamento ou bloqueio em vias)
    \item Semáforos
    \item Acidentes de trânsitos
    \item Greves e movimentações
    \item Incapacidade de registrar uma velocidade média (visto que não há registro de placas)
    \item Carros andando em velocidades ilegais na falta de presença de fiscalização eletrônica.
\end{itemize}
Todos esses eventos podem influenciar nos resultados obtidos e dependendo da escala do evento, podem causar uma disparidade significante entre a predição e a realidade.

% Classe de aplicação
Além disso, o modelo está sendo treinado com dados de uma só localidade, recolhidos de sensores instalados em locais específicos. Logo, os resultados podem não ser os mesmos quando a metodologia for usada em outros locais. O modo como os cidadãos dirigem e como as vias são feitas também vão influenciar nisso.

Para a previsão perfeita seriam necessárias muitas variáveis as quais não estão disponíveis. Então vamos tentar simplificar o problema e tentar ver as tendências.