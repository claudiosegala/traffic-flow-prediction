\section{Motivação}

% TODO: mencionar manifestações e greves como mais um fator de causa de congestionamento
O volume de carros tem aumentado ao longo dos últimos anos e as vias públicas não tem conseguido acompanhar este crescimento \cite{sindipecas_2019}. Com isso, os congestionamentos tendem a piorar, aumentando a emissão de gases poluentes na atmosfera, poluição sonora e o tempo gasto pelos motoristas no trânsito, diminuindo assim a qualidade de vida do habitantes de uma cidade.

Um dos meios de amenizar o congestionamento é tentar prever quando ele irá acontecer. Uma vez que os cidadãos saibam que um engarrafamento irá ocorrer, eles podem se planejar melhor e tomar rotas alternativas, distribuindo melhor os carros na malha viária.

Em razão do exposto anteriormente, diversas pesquisas na área em questão foram desenvolvidas. As metodologias usadas nessas pesquisas giram em torno de predições baseadas em aprendizagem profunda \cite{Xiaolei_2015} \cite{Zainab_2018} \cite{he2013improving} \cite{wang_2018} \cite{lv_6894591} e predições heurísticas \cite{levin_forecast_1980} \cite{hamed_prediction_1995} \cite{voort_series_1996}.As metodologias heurísticas utilizam de regressões, tais como \acrfull{ARIMA}. Já as de aprendizagem profunda utilizam de arquiteturas mais complexas, tais como \acrfull{SLSTM} e \acrfull{SAE}. 
%Sendo que o estado da arte é \acrshort{SLSTM}.

Essas pesquisas tem como objetivo gerar uma série temporal sobre o comportamento de uma via nos próximos t minutos, baseados em seu comportamento passado. Apesar de utilizarem os mais diversos e sofisticados métodos, nenhum destes trabalhos se dispõe do \acrfull{GAN}, uma arquitetura muito utilizada para gerar dados que se assemelham a outros previamente vistos.

\section{Justificativa}

\acrshort{GAN} é uma arquitetura de aprendizagem profunda que vem trazendo resultados promissores em diversas áreas \cite{christian}. Ela é uma arquitetura excelente para reproduzir os dados de entrada, aprendendo a distribuição dos mesmos. Visto que queremos gerar uma série temporal, \arcshort{GAN} deve ser capaz de aprender a distribuição das séries temporais que representam o passado em t minutos para gerar uma série temporal que corresponda a uma previsão de t minutos no futuro.


Essa arquitetura, no entanto, trabalha com dados independentes entre si, mas existem variações capazes de lidar com séries temporais, que são dependentes entre si, como mostrado em \cite{zhou_2018}, \cite{banushev_2019} e \cite{esteban2017real}, que obtiveram resultados promissores.

\section{Objetivos}

Este trabalho tem como objetivo propor uma arquitetura capaz de realizar predições de tráfego de curto e médio período (5, 10, 15, 30, 60, 120 e 150 minutos) utilizando GAN.

Para tal, serão utilizados os dados de fiscalização eletrônica do \acrfull{DF} para treinar uma variação da arquitetura \acrshort{GAN}, usando uma \acrshort{LSTM} como gerador e como discriminador. Sendo assim, os objetivos específicos desta proposta são:

\begin{itemize}
    \item Implementar uma arquitetura capaz de gerar séries temporais que simulem os próximos t minutos dado um intervalo de série temporal.
    \item Treinar a arquitetura de forma a evitar erros comuns no treinamento de \acrshort{GAN}, como \textit{overfitting}.
    \item Avaliar o desempenho desta metodologia se comparada ao estado da arte nos períodos propostos.
\end{itemize}

\section{Resultados Esperados}
% LATER: adicionar métricas de acurácia

É esperado que a arquitetura proposta tenha um desempenho similar ou superior ao estado da arte, que também usa de \acrshort{LSTM}. Para tal, serão usados três métricas, sendo elas \acrfull{MAE}, \acrfull{RMSE} e \acrfull{MRE}. 

\begin{equation}
\acrshort{MAE} = \frac{1}{n} \times \sum_{i=1}^{n} \quad \abs{result_i - expected_i}
\end{equation}

\begin{equation}
\acrshort{MRE} = \frac{1}{n} \times \sum_{i=1}^{n} \quad \frac{\abs{result_i - expected_i}}{result_i}
\end{equation}

\begin{equation}
\acrshort{RMSE} = \sqrt{ \frac{1}{n} \times \sum_{i=1}^{n} \quad (result_i - expected_i) ^ 2}
\end{equation}
    
\section{Limitações}
Para se fazer a previsão serão desconsiderados certos cenários em que a captura e o relacionamentos dos dados obtidos são muito imprecisos e/ou pontuais. Esses eventos podem influenciar nos resultados obtidos e dependendo da escala do evento, podem causar uma disparidade significante entre a predição e a realidade. Entre eles estão:
\begin{itemize}
    \item Falhas no equipamento ou bloqueio em vias
    \item Semáforos
    \item Acidentes de trânsitos
    \item Greves e movimentações
    \item Incapacidade de registrar uma velocidade média (visto que não há registro de placas)
    \item Carros transitando em velocidades ilegais na falta de presença de fiscalização eletrônica.
\end{itemize}
Além disso, o modelo será treinado com dados de uma só cidade, recolhidos de sensores instalados em locais específicos. Logo, os resultados podem não ser os mesmos quando a metodologia for utilizada em outras localidades. O modo como os cidadãos dirigem e como as vias são feitas também vão influenciar. Para a previsão perfeita seriam necessárias muitas variáveis, as quais não estão disponíveis.

\section{Metodologia}
Visto o objetivo dessa proposta, o trabalho será divido em 4 partes:

\begin{itemize}
    \item Coleta de dados: Adquirir dados de fiscalização eletrônica local, no caso \acrshort{DETRAN};
    \item Pré-processamento: verificar e retirar dados que possuem inconsistências, ou que não importam no treinamento da arquitetura;
    \item Implementação: programar todas as arquiteturas necessárias para a comparação e treinar utilizando os mesmos dados;
    \item Avaliação: utilizar das métricas capturadas durante a execução de ambas arquiteturas para avaliar o desempenho.
\end{itemize}

%Porém existem pesquisas que tentam aprimorar o estado da arte tentando coletar e utilizar das variáveis aleatórias provindas de redes sociais como o Twitter \cite{he2013improving} ou utilizam de dados meteorológicos \cite{wang_2018}. Mas no geral, essas variáveis são ignoradas.

\section{Hipótese}
% TALVEZ: criar mais hipóteses que sejam mais fáceis de provar
% TALVEZ: comparar as duas quando se adiciona chuva
% TALVEZ: tentar prever até um dia na frente
% TALVEZ: avaliar a importância dos atributos que eu tenho
% TALVEZ: pegar mais fontes de dados e ver como reage
% TALVEZ: avaliar melhores janelas de tempo para o algoritmo

Queremos verificar se a arquitetura \acrshort{GAN} proposta nesse trabalho é capaz de gerar séries temporais para previsão do tráfego melhor que o estado da arte, isto é, uma arquitetura \acrshort{SLSTM}.