Predição de fluxo de veículos é uma tarefa complexa com vários fatores desafiadores como a escalabilidade, a oscilação do fluxo e os fatores randômicos. A literatura, em sua maioria, foca em vias livres. Visto isso, este trabalho aborda cinco modelos de aprendizagem de máquina comumente usados para resolver o problema com vias livres (\textit{Long Short Term Memory Neural Network}, \textit{Gated Recurrent Unit Neural Network}, \textit{Recurrent Neural Network}, \textit{Support Vector Machine} e \textit{Random Forest}) para verificar se os modelos conseguem ter uma performance satisfatória em vias urbanas, mais especificamente cruzamentos. Esta pesquisa ainda está em andamento, mas os modelos já se mostraram significativamente superiores que as bases de comparação.