% TODO: ver com o Li se faz sentido escrever sobre cidades inteligentes só na introdução. Se fizer, escrever o primeiro parágrafo falando como que cidades inteligentes são a tendência e como um dos maiores problemas é congestionamento. Que elas querem ser mais limpas

A frota  de  veículos no Brasil e no mundo tem aumentado rapidamente ao longo  dos últimos anos. Como pode ser visto em \cite{G1}, em 2018, o Brasil já apresentava 1 carro para cada 4 habitantes. Isso equivale, aproximadamente, a 50 milhões de automóveis em circulação. As vias públicas não tem conseguido acompanhar este crescimento e acomodar todo este volume. Brasília, por exemplo, que é considerada uma cidade bem planejada, foi projetada para suportar uma frota de no máximo 2 milhões de veículos, porém já é obrigada a conciliar 3.5 milhões, como pode ser visto em \cite{detran_2018}, e começa a sofrer com os congestionamentos. Infelizmente, Brasília não é uma exceção, todos os grandes centros urbanos do país estão sendo afetados pelas mazelas do crescimento do setor automobilístico, e pensar em soluções para estes problemas é mais que necessário.

% TODO: remover piada
Em resposta a essa tendência, as metrópoles tem utilizado da tecnologia para gerenciar melhor o trânsito em suas vias, ao passo que os motoristas tem feito uso cada vez mais intenso de aplicativos como \textit{Waze} e \textit{Google Maps} para escapar de congestionamentos e descobrir os caminhos mais rápidos até seus destinos. Porém, se essas tecnologias e aplicações não se preocuparem em como o fluxo da via estará nos próximos minutos, isso pode levar a erros de cálculo da melhor rota. Se não temos como prever o futuro, como podemos então nos planejar melhor ? Utilizando-se de técnicas de aprendizagem de máquina é possível aprender como uma via tende a se comportar, podendo assim estimar como o seu fluxo de veículos estará em alguns minutos.

%Neste contexto de expansão urbana e rápido crescimento da frota de veículos, surgiu o conceito de cidades inteligentes (referencia para smart cities), que consiste em aplicar tecnologia no gerenciamento das áreas urbanas que afetam a população e sua saúde, visando melhorá-las. Isso inclui o gerenciamento do tráfego urbano e sua otimização. 

\section{Motivação}
% TODO: deixar frase inicial mais coerente
Como dito anteriormente, os grandes centros urbanos não estão conseguindo acompanhar a constante expansão do volume de carros. Por conta disso, congestionamentos tendem a se tornar mais frequentes e mais severos, aumentando a emissão de gases poluentes na atmosfera, contribuindo para a poluição sonora e aumentando o tempo gasto pelos motoristas no trânsito, impactando de maneira negativa na qualidade de vida da cidade e seus habitantes. % Todos estes efeitos negativos vão de encontro ao conceito de cidades inteligentes.

Neste contexto, surgiram várias pesquisas na área de mobilidade urbana e transporte que visam contornar estes problemas. Dentre essas pesquisas, um dos meios encontrados de amenizar o congestionamento é tentar prever quando ele irá acontecer. Uma vez que os cidadãos saibam que um engarrafamento irá ocorrer, eles podem se planejar melhor e tomar rotas alternativas, distribuindo de maneira mais eficiente os carros na malha viária. Já organizações governamentais poderiam usar os dados para ter uma noção em tempo real de como está o fluxo pela cidade, podendo agir redirecionando vias e reprogramando semáforos.

A literatura está repleta de artigos de predição de fluxo utilizando aprendizagem de máquina. Porém, a maioria desses artigos coletam dados e fazem suas análises em cima de rodovias expressas. Nessas não há intersecções, ou barreiras semafóricas. Nosso estudo tem como desafio realizar a predição de fluxo em cenários que se assemelhem mais a centros urbanos, onde as vias públicas possuem cruzamentos e radares de velocidade que impactam de maneira significativa no comportamento do fluxo de veículos.

\section{Objetivo}
% TODO: melhorar muito

Como já foi dito, este trabalho busca realizar a predição de fluxo de veículos em vias urbanas. Para tal, usaremos os métodos mais explorados na literatura. Como tais métodos são comumente utilizados para o cálculo de fluxo em vias livres, este trabalho tem como objetivo propor uma análise do comportamento destes mesmos métodos sob novas condições (vias urbanas com cruzamentos) e comparar a eficácia da predição de cada um deles.
 
\section{Justificativa}
% TODO: mostrar um gráfico mostrando o crescimento dos congestionamento
% TODO: adicionar referência mostrando a quantidade de congestionamento diferentes.
% TODO: mencionar algum artigo que fala que o google maps utiliza de sua localização quando sente que vc está dirigindo


% TODO: Criar uma justificativa mais plausível

Congestionamentos têm ocorrido com mais frequência e aumentado o tempo de duração ao longo dos anos. 


\section{Estrutura da Monografia}

Esta monografia está estruturado em 7 capítulos. Retirando o capítulo corrente de introdução, ainda temos:

\begin{itemize}
    \item \textbf{Capítulo 2:} teoria necessária para o entendimento do trabalho;
    \item \textbf{Capítulo 3:} exposição dos trabalhos similares e relacionados ao nosso;
    \item \textbf{Capítulo 4:} análise dos tipos de dados normalmente utilizados e do que vamos utilizar;
    \item \textbf{Capítulo 5:} exposição detalhada da metodologia de comparação e das escolhas feitas nesse trabalho;
    \item \textbf{Capítulo 6:} mostra dos resultados obtidos e análise dos mesmos;
    \item \textbf{Capítulo 7:} conclusão do nosso trabalho e como nosso trabalho poderia evoluir;
\end{itemize}

% TALVEZ ----------------------------------

% TODO?: Falar que o método utilizado por algumas empresas chega a ser invasivo (Google Maps)

% TODO?: Falar algum projeto maior que uma das etapas exige previsão de fluxo (Falar sobre controle de semáforo dinâmico e da possibilidade de uma central de controle que permite visualizar o tráfego futuro em tempo real que poderia auxiliar na tomada de decisão)
