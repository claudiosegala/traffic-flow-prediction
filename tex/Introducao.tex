
A frota  de  carros no Brasil e no mundo  tem  aumentado rapidamente ao  longo  dos  últimos anos. Como pode ser visto em (citar fonte que mostra a quantidade de carros para habitantes), em 2018, o Brasil ja apresentava 1 carro para cada 4 habitantes, isso equivale, aproximadamente, a 50 milhões de carros em circulação. As vias públicas não tem conseguido acompanhar este crescimento. Brasília, por exemplo, que é considerada uma cidade bem planejada, foi projetada para suportar uma frota de no máximo 2 milhões de carros, porém já é obrigada a conciliar 3,5 milhões, como pode ser visto em (link para referencia com a quantidade de carros no df nos ultimos 10 anos).

Neste contexto de expansão urbana e rápido crescimento da frota de veículos, surgiu o conceito de cidades inteligentes (referencia para smart cities), que consiste em aplicar tecnologia no gerenciamento das áreas urbanas que afetam a população e sua saúde, visando melhorá-las. Isso inclui o gerenciamento do tráfego urbano e sua otimização. 

Com  isso,  os  congestionamentos  tendem  a piorar, aumentando a emissão de gases poluentes na atmosfera,poluição sonora e o tempo gasto pelos motoristas no trânsito,diminuindo  assim  a  qualidade  de  vida  do  habitantes  de  uma cidade.Um  dos  meios  de  amenizar  o  congestionamento  é  tentar prever  quando  ele  irá  acontecer.  Uma  vez  que  os  cidadãos saibam  que  um  engarrafamento  irá  ocorrer,  eles  podem  se planejar melhor e tomar rotas alternativas, distribuindo melhor os  carros  na  malha  viária.  Já  organizações  governamentais poderiam  usar  os  dados  para  ter  uma  noção  em  tempo  real de  como está  o  fluxo pela  cidade,  podendo agir  antes de  um engarrafamento acontecer.A literatura está saturada com artigos de predição de fluxo utilizando aprendizagem de máquina. Porém, a maioria desses artigos  coletam  dados  e  fazem  suas  análises  em  cima  de rodovias  expressas.  Nessas  não  há  intersecções,  ou  barreiras semafóricas.  Nosso  estudo  tem  como  desafio  aplicar  estes modelos  em  vias  urbanas,  onde  há  cruzamentos  e  radares  de velocidade que impactam de maneira significativa no comportamento do fluxo de veículos, e verificar a eficácia dos modelos utilizados na literatura nestas novas condições.

Mais especificamente, este trabalho tem como objetivo fazer a  predição  de  fluxo  em  cruzamento  utilizando  de  modelos tradicionais e de aprendizagem profunda. Para assim verificar se possuem uma performance satisfatória para cruzamentos.
\section{Motivação}



\section{Objetivo}

