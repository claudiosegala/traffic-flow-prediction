%%%%%%%%%%%%%%%%%%%%%%%%%%%%%%%%%%%%%%%%
% Classe do documento
%%%%%%%%%%%%%%%%%%%%%%%%%%%%%%%%%%%%%%%%

% Opções:
%  - Graduação: bacharelado|engenharia|licenciatura
%  - Pós-graduação: [qualificacao], mestrado|doutorado, ppca|ppginf

\documentclass[bacharelado]{UnB-CIC}%

\usepackage{pdfpages}% incluir PDFs, usado no apêndice
\usepackage{commath} % inclui expressões matemáticas
\usepackage{booktabs}

%%%%%%%%%%%%%%%%%%%%%%%%%%%%%%%%%%%%%%%%
% Informações do Trabalho
%%%%%%%%%%%%%%%%%%%%%%%%%%%%%%%%%%%%%%%%
\orientador{\prof \dr Li Weigang}{CIC/UnB}%
\coorientador{\prof \dr Geraldo Pereira Rocha Filho}{CIC/UnB}
\coordenador{\prof \dr Edison Ishikawa}{Instituto Militar de Engenharia}%
\diamesano{7}{julho}{2019}%


\membrobanca{\prof \dr Li Weigang}{CIC/UnB}%
\membrobanca{\prof \dr Geraldo Pereira Rocha Filho}{CIC/UnB}

\autor{Claudio}{S. Rodrigues Silva Filho}%
\coautor{Pedro Henrique}{S. Gonzaga}%

\titulo{Predição de Fluxo de Veículos em Cruzamentos: Comparando os Algoritmos Tradicionais com Aprendizagem Profunda}%

\palavraschave{cruzamento de vias, aprendizagem de máquina, predição, fluxo de veículos,  redes neurais, tráfego}
\keywords{roads crossing, machine learning, forecasting, prediction, vehicle flow, neural networks, traffic}%

\newcommand{\unbcic}{\texttt{UnB-CIC}}%
 
\usepackage{float} 
\usepackage{color}
\usepackage{listings}

%%%%%%%%%%%%%%%%%%%%%%%%%%%%%%%%%%%%%%%%
% Texto
%%%%%%%%%%%%%%%%%%%%%%%%%%%%%%%%%%%%%%%%

\begin{document}
    \capitulo{Introducao}{Introdução}%
    \capitulo{Fundamentacao_Teorica}{Fundamentação Teórica}%
    \capitulo{Trabalhos_Relacionados}{Trabalhos Relacionados}%
    \capitulo{Analise_Dados}{Análise dos Dados}%
    \capitulo{Metodologia}{Metodologia}%
    \capitulo{Resultados}{Resultados}
    \capitulo{Conclusao}{Conclusão}

    %\apendice{Apendice_Fichamento}{Fichamento de Artigo Científico}%
    \anexo{Attachment_Code}{Código Utilizado}
    \anexo{Attachment_Flows}{Fluxos Registrados}
\end{document}%


% CHECK ON FINAL VERSION ---------

% - checar se tudo que ta em inglês está em itálico
% - checar se tudo que é sigla está em acrshort e acrfull
% - checar se a primeira aparição de todas as siglas está acrfull
% - checar se estamos usando média móvel como nome do modelo de base
% - rodar em mil máquinas de plágio para evitar falso positivo tardio
% - checar se tem em anexo o codigo 
% - checar se o link para os dados está funcionando
% - https://cs.stanford.edu/people/widom/paper-writing.html
% - checar se todas as imagens estão sendo citadas

% RESOURCES -----------------------

% Paramêtrico e Não-Paramétrico (https://www.youtube.com/watch?v=UgwUi8fu0CY)
