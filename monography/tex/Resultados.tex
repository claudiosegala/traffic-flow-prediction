% TODO: Falar que utilizamos o hit rate para ver se nossos modelos eram capazes de pelo menos prever quando que vai diminuir e quando que vai aumentar

% TODO: Melhorar os gráficos

% TODO?: Adicionar gráfico que mostra o tempo de predição

% TODO: Adicionar gráfico de como os modelos se comportam com o aumento do tamanho da janela

% TODO: Adicionar gráfico de como os modelos se comportam com o aumento da visão do passado

% TODO: Adicionar gráfico de como os modelos se comportam com o aumento e diminuição do intervalo de fluxo 

Este capítulo tem como objetivo apresentar uma discussão acerca dos resultados dos experimentos. Primeiramente, serão discutidos como os modelos reagem à variação dos principais parâmetros, quantidade de divisões, passado visível e tamanho do intervalo do fluxo. Subsequentemente, serão apresentados os resultados de como os modelos reagem para a escolha dos hiper-parâmetros selecionados. Por fim, serão analisados os resultados das previsões de fluxo de cada modelo nos curtos prazos de 15, 30, 45 e 60 minutos.

\section{Resultados das Escolhas de Parâmetros}

Após o tratamento dos dados, foram realizados diversos testes para definir os melhores parâmetros de cada modelo. Devido a limitações de hardware os parâmetros foram considerados independentes. Ou seja, cada um deles possui um valor padrão e será analisado como esse parâmetro se comporta dado a fixação dos outros. Além disso, as otimizações estão sendo feitas no pior caso da predição, isto é, na predição de 1h no futuro. Na lista abaixo podem ser vistos os parâmetro que foram levados em consideração nos estudos de caso deste trabalho.

\begin{enumerate}
	\item \textbf{Número de divisões do conjunto de dados} (\textit{Blocking}): Número de divisões utilizadas no blocking para criar mais subconjuntos de dados. Os valores testados foram: 1,2,4,8. Para cada subconjunto, a porcentagem utilizada para treinamento e teste foi a mesma: 80\% para treinamento e 20\% para testes. O valor padrão utilizado foi de 4 divisões.
	\item \textbf{Passado Visível}: Parâmetro que define qual a janela de tempo no passado que será visível no treinamento do modelo. O valor inicial considerado foi de 480 minutos. Os valores de 60, 120, 240 e 480 minutos também foram testados.
	\item \textbf{Tamanho do intervalo do fluxo}: Parâmetro que define o intervalo de tempo no qual será acumulado a quantidade de veículos para o cálculo do fluxo. O valor inicial de teste foi 2,5 minutos. Foram realizados testes para as variações de 5 e 7,5 minutos também (ou 150, 300 e 450 segundos).
\end{enumerate}

\subsection{Número de Divisões do Conjunto de Dados}

Na Figura \ref{figure:res_split} pode ser observado que todos os modelos apresentaram uma performance melhor que as bases de comparação. Os resultados também são um indicativo que não é necessário um tamanho muito grande de treino para que os modelos se ajustem a distribuição do conjunto de dados. No caso, o melhor número de divisão do conjunto de dados para a maioria os modelos foi  8, como pode ser observado na Tabela \ref{table:res_split}. Como o tamanho do conjunto de dados é de 52.992 (intervalo de fluxo padrão foi de 150 segundos), os modelos foram capazes de se ajustar com apenas 5300 dados (aproximadamente), o que seria equivalente a um pouco mais de 1 semana. Os únicos modelos que não apresentaram os melhores resultados com o número de divisões igual a 8 foram \textit{\acrshort{SVM}} e \textit{\acrshort{GRU}} que utilizaram a versão A do conjunto de dados. Porém, mesmo estes modelos tiveram diferenças pouco significativas se comparados com os seus resultado de 8 divisões. Vale notar também que a maioria dos modelos apresentou uma piora nas predições quando fora utilizado 4 como número de divisão, o que seria equivalente a duas semanas.

\begin{figure}[htbp]
    \centering
    \includegraphics[scale=0.8]{monography/img/number_of_splits_for_training_comparison_rmse.png}
    \label{figure:res_split}
    \caption[Resposta dos modelos à variação do número de divisões do conjunto de dados]{Resposta dos modelos à variação do número de divisões do conjunto de dados}
\end{figure} 

\begin{table}[htbp]
    \begin{tabular*}{\linewidth}{@{\extracolsep{\fill}}lllll}
    \toprule
     & 
    \multicolumn{1}{l}{\textbf{1}} & 
    \multicolumn{1}{l}{\textbf{2}} &
    \multicolumn{1}{l}{\textbf{4}} &
    \multicolumn{1}{l}{\textbf{8}} \\
    \midrule
    \textbf{Moving Av.} & 9.596 $\pm$ 0.000 & 9.399 $\pm$ 0.008 & 9.098 $\pm$ 0.740 & \textbf{9.044} $\pm$ 0.881
    \\
    \midrule
    \textbf{Naive} & 9.413 $\pm$ 0.000 & 8.492 $\pm$ 0.777 & 8.026 $\pm$ 0.868 & \textbf{7.863} $\pm$ 0.889
    \\
    \midrule
    \textbf{RF A} & 4.924 $\pm$ 0.000 & 4.469 $\pm$ 0.254 & 4.569 $\pm$ 0.739 & \textbf{4.439} $\pm$ 0.513
    \\
    \midrule
    \textbf{RF B} & 4.997 $\pm$ 0.000 & 4.504 $\pm$ 0.240 & 4.610 $\pm$ 0.745 & \textbf{4.508} $\pm$ 0.501
    \\
    \midrule
    \textbf{SVM A} & 4.836 $\pm$ 0.000 & \textbf{4.555} $\pm$ 0.160 & 4.669 $\pm$ 0.736 & 4.556 $\pm$ 0.494
    \\
    \midrule
    \textbf{SVM B} & 5.205 $\pm$ 0.000 & \textbf{4.912} $\pm$ 0.042 & 5.011 $\pm$ 0.618 & 5.030 $\pm$ 0.513
    \\
    \midrule
    \textbf{LSTM A} & 6.418 $\pm$ 0.000 & 5.251 $\pm$ 0.279 & 5.239 $\pm$ 0.794 & \textbf{5.123} $\pm$ 0.672
    \\
    \midrule
    \textbf{LSTM B} & 6.412 $\pm$ 0.000 & 5.197 $\pm$ 0.156 & 5.261 $\pm$ 0.830 & \textbf{4.996} $\pm$ 0.696
    \\
    \midrule
    \textbf{GRU A} & 6.366 $\pm$ 0.000 & 5.312 $\pm$ 0.014 & \textbf{5.137} $\pm$ 0.702 & 5.143 $\pm$ 0.516
    \\
    \midrule
    \textbf{GRU B} & 6.545 $\pm$ 0.000 & 5.190 $\pm$ 0.030 & 5.218 $\pm$ 0.686 & \textbf{4.963} $\pm$ 0.643
    \\
    \bottomrule
    \end{tabular*}
    \label{table:res_split}
    \caption{Resultados da Comparação entre os números de divisões. Melhores resultados de cada modelo em negrito.}
\end{table}

\subsection{Passado Visível}

% TODO: reference the seeable_past_time

A respeito da variação no quanto do passado o modelo tem acesso, é perceptível na Figura \ref{figure:res_past} que quanto mais acesso ao passado o modelo tiver, melhor será sua predição. Percebe-se também que o melhor valor para a quantidade de acesso do passado, dentre os testados, são 480 minutos (ou 8 horas). Porém, é importante ressaltar que mais acesso ao passado também implica em uma quantidade de treino maior, ou seja, maior custo computacional, especialmente para modelos de aprendizagem profunda. Dito isso, e considerando que a escala do eixo X é exponencial (Passado visível: 60, 120, 240, 480), pode-se que concluir que para conseguir resultados ainda melhores, seria necessário aumentar de maneira significativa a quantidade de passado visível, o que, talvez, não valesse o aumento do custo computacional, já que, para cada aumento exponencial do eixo X, a qualidade da predição melhora apenas de maneira linear. A respeito das bases de comparação, também é claro que o modelo \textit{Naive} não é afetado pela quantidade de tempo de passado visível, mas ainda assim há uma leve mudança no conjunto de dados, pois ao se mudar este parâmetro, o último valor recebido pelo modelo \textit{Naive} é mudado também, afetando suas predições e gerando as diferenças vistas na Tabela \ref{table:res_past}. Já o modelo \textit{Moving Average} é o único que piora com o tempo. 


\begin{figure}[htbp]
    \centering
    \includegraphics[scale=0.8]{monography/img/seeable_past_for_training_comparison_rmse.png}
    \label{figure:res_past}
    \caption[Resposta dos modelos à variação dos valores de passado visível]{Resposta dos modelos à variação dos valores de passado visível}
\end{figure}
 
\begin{table}[htbp]
    \begin{tabular*}{\linewidth}{@{\extracolsep{\fill}}lllll}
    \toprule
     & 
    \multicolumn{1}{l}{\textbf{60}} & 
    \multicolumn{1}{l}{\textbf{120}} &
    \multicolumn{1}{l}{\textbf{240}} &
    \multicolumn{1}{l}{\textbf{480}} \\
    \midrule
    \textbf{Moving Av.} & \textbf{6.485} $\pm$ 0.627 & 6.836 $\pm$ 0.649 & 7.626 $\pm$ 0.684 & 9.098 $\pm$ 0.740
    \\
    \midrule
    \textbf{Naive} & 8.120 $\pm$ 0.869 & \textbf{8.102} $\pm$ 0.873 & 8.065 $\pm$ 0.887 & 8.026 $\pm$ 0.868 
    \\
    \midrule
    \textbf{RF A} & 5.089 $\pm$ 0.758 & 4.910 $\pm$ 0.735 & 4.741 $\pm$ 0.707 & \textbf{4.569} $\pm$ 0.739 
    \\
    \midrule
    \textbf{RF B} & 5.069 $\pm$ 0.753 & 4.918 $\pm$ 0.728 & 4.766 $\pm$ 0.705 & \textbf{4.610} $\pm$ 0.745 
    \\
    \midrule
    \textbf{SVM A} & 5.230 $\pm$ 0.700 & 5.118 $\pm$ 0.706 & 5.015 $\pm$ 0.691 & \textbf{4.669} $\pm$ 0.736 
    \\
    \midrule
    \textbf{SVM B} & 5.412 $\pm$ 0.627 & 5.318 $\pm$ 0.624 & 5.129 $\pm$ 0.586 & \textbf{5.011} $\pm$ 0.618 
    \\
    \midrule
    \textbf{LSTM A} & 5.675 $\pm$ 0.814 & 5.602 $\pm$ 0.713 & 5.409 $\pm$ 0.750 & \textbf{5.260} $\pm$ 0.769 
    \\
    \midrule
    \textbf{LSTM B} & 5.497 $\pm$ 0.563 & 5.463 $\pm$ 0.573 & 5.343 $\pm$ 0.731 & \textbf{5.220} $\pm$ 0.763 
    \\
    \midrule
    \textbf{GRU A} & 5.413 $\pm$ 0.702 & 5.417 $\pm$ 0.650 & 5.230 $\pm$ 0.610 & \textbf{5.191} $\pm$ 0.742 
    \\
    \midrule
    \textbf{GRU B} & 5.364 $\pm$ 0.631 & 5.322 $\pm$ 0.710 & 5.221 $\pm$ 0.747 & \textbf{5.087} $\pm$ 0.679
    \\
    \bottomrule
    \end{tabular*}
    \label{table:res_past}
    \caption{Resultados da Comparação entre as quantidade de passados visíveis. Melhores resultados de cada modelo em negrito.}
\end{table}
 
\subsection{Tamanho do Intervalo do Fluxo}

Por último, pode-se notar na Figura \ref{figure:res_flow} que o tamanho do intervalo do fluxo adotado impacta de maneira significativa a qualidade das predições dos modelos. Esse efeito pode ser justificado pela quantidade de dados resultantes de cada tamanho do intervalo de fluxo, visto que quanto maior o intervalo, menor a quantidade de dados. Por exemplo, são 52.992 dados para 2,5 minutos (150 segundos) e 17.664 para 7.5 minutos (450 segundos). O que torna mais extenso o tempo de treinamento. Dos valores testados, 2.5 minutos é o melhor tamanho de intervalo de fluxo para todos os modelo, como pode ser visto na Tabela \ref{table:res_flow}.

\begin{figure}[H]
    \centering
    \includegraphics[scale=0.8]{monography/img/flow_interval_for_training_comparison_rmse.png}
    \label{figure:res_flow}
    \caption{Resposta dos modelos à variação dos valores de intervalo de fluxo}
\end{figure}

\begin{table}[H]
    \begin{tabular*}{\linewidth}{@{\extracolsep{\fill}}llll}
    \toprule
     & 
    \multicolumn{1}{l}{\textbf{150}} & 
    \multicolumn{1}{l}{\textbf{300}} &
    \multicolumn{1}{l}{\textbf{450}} \\
    \midrule
    \textbf{Moving Av.} & \textbf{9.098} $\pm$ 0.740 & 16.228 $\pm$ 0.875 & 23.233 $\pm$ 1.455
    \\
    \midrule
    \textbf{Naive} & \textbf{8.026} $\pm$ 0.868 & 12.494 $\pm$ 2.059 & 13.954 $\pm$ 1.677
    \\
    \midrule
    \textbf{RF A} & \textbf{4.569} $\pm$ 0.739 & 6.472 $\pm$ 0.464 & 8.582 $\pm$ 1.416
    \\
    \midrule
    \textbf{RF B} & \textbf{4.610} $\pm$ 0.745 & 6.525 $\pm$ 0.383 & 8.543 $\pm$ 1.225
    \\
    \midrule
    \textbf{SVM A} & \textbf{4.669} $\pm$ 0.736 & 6.714 $\pm$ 0.327 & 9.178 $\pm$ 0.955
    \\
    \midrule
    \textbf{SVM B} & \textbf{5.011} $\pm$ 0.618 & 7.361 $\pm$ 0.381 & 9.843 $\pm$ 0.360
    \\
    \midrule
    \textbf{LSTM A} & \textbf{5.269} $\pm$ 0.803 & 7.356 $\pm$ 0.601 & 9.976 $\pm$ 0.802
    \\
    \midrule
    \textbf{LSTM B} & \textbf{5.188} $\pm$ 0.715 & 7.246 $\pm$ 0.631 & 9.929 $\pm$ 0.980
    \\
    \midrule
    \textbf{GRU A} & \textbf{5.183} $\pm$ 0.679 & 7.336 $\pm$ 0.602 & 9.993 $\pm$ 1.034
    \\
    \midrule
    \textbf{GRU B} & \textbf{5.164} $\pm$ 0.758 & 7.106 $\pm$ 0.663 & 9.444 $\pm$ 1.521
    \\
    \bottomrule
    \end{tabular*}
    \label{table:res_flow}
    \caption{Resultados da Comparação entre os intervalos de fluxo. Melhores resultados de cada modelo em negrito.}
\end{table}


\section{Comparação dos Resultados com \textit{Tuning} dos Modelos}

O próximo passo foi escolher os melhores valores de parâmetros para cada modelo. Para alcançar tal objetivo, foi feito um \textit{Grid Search} durante as comparações dos resultados das predições de curto prazo. Abaixo podem ser vistos os resultados com \textit{Tuning} dos parâmetros e sem \textit{Tuning} dos parâmetros.
