% TODO: Falar que utilizamos o hit rate para ver se nossos modelos eram capazes de pelo menos prever quando que vai diminuir e quando que vai aumentar

% TODO: Melhorar os gráficos

% TODO?: Adicionar gráfico que mostra o tempo de predição

% TODO: Adicionar gráfico de como os modelos se comportam com o aumento do tamanho da janela

% TODO: Adicionar gráfico de como os modelos se comportam com o aumento da visão do passado

% TODO: Adicionar gráfico de como os modelos se comportam com o aumento e diminuição do intervalo de fluxo 



Após o tratamento dos dados, foram realizados diversos testes para definir os melhores parâmetros de cada modelo. Na lista abaixo podem ser vistos os parâmetro que foram levados em conta nos nossos estudos de caso.

\begin{enumerate}
	\item 
	\item Two
	\item Three
\end{enumerate}

\begin{table}[h]
    \caption{Comparação das predições para os modelos univariados com e sem normalização}
    \label{table:RmseComparison}
    \begin{center}
    \begin{tabular}{ccccccc}
    \hline
    \multicolumn{1}{l}{\textbf{Modelo}} & \multicolumn{1}{l}{\textbf{RMSE}} & \multicolumn{1}{l}{\textbf{NRMSE}} & \multicolumn{1}{l}{\textbf{MAE}} \\
    \hline
    SVM & 4.14 & 0.47 & 2.82  \\
    Mean & 6.20 & 0.71 & 4.55 \\
    Random Guess & 18.23 & 2.11 & 14.93\\
    RNN & 4.29 & 0.49 & 2.96 \\ 
    GRU & 4.48 & 0.51 & 3.11  \\ 
    LSTM Mult & 4.99 &  0.57 & 3.58  \\ 
    LSTM Uni & 4.74 &  0.54 & 3.37  \\ 
    Random Forest & 4.17 & 0.48 & 2.91 \\
    \hline
    \end{tabular}
    \end{center}
\end{table}


A primeira comparação feita foi entre os resultados das predições dos modelos com o valores de entrada normalizados e não normalizados. Para os modelos uni-variados, normalizamos o valor do fluxo utilizando a biblioteca \textit{Sklearn}, onde o menor valor é mapeado como 0 e o maior valor como 1. Segundo [Citar fonte que mostra que dados normalizados tendem a ter um melhor resultado], valores normalizados entre 0 e 1 tendem a ter um melhor resultado para algoritmos que utilizam funções de ativação sigmoidais, além de evitar que o modelo fique enviesado para os valores com maiores ordens de grandeza. Porém, como pode ser visto na tabela 1, no caso do \acrfull{LSTM}, o \acrfull{RMSE} ficou levemente pior para as predições utilizando valores normalizados em seu treinamento. Isso pode ter acontecido pois o valor do intervalo do fluxo oriundo dos nossos dados não era tão grande, sendo o menor valor 0 e o maior valor 60. Com um intervalo pequeno como esse, a normalização não teve um impacto tão grande, pois 0 e 60 estão apenas a uma ordem de grandeza de diferença.

O mesmo raciocínio pode ser utilizado para explicar a tabela 2, onde podemos observar que a normalização também não trouxe grandes benefícios aos modelos multivariados. Novamente, a ordem de grandeza entre os dados de entrada e entre os seus diferentes tipos (velocidade, fluxo) é bastante pequena, tornando a normalização pouco eficaz também nos casos dos modelos multivariados.

\begin{table}[H]
    \caption{Comparação das predições para os modelos multivariados com e sem normalização}
    \label{table:RmseComparison}
    \begin{center}
    \begin{tabular}{ccccccc}
    \hline
    \multicolumn{1}{l}{\textbf{Modelo}} & \multicolumn{1}{l}{\textbf{RMSE}} & \multicolumn{1}{l}{\textbf{NRMSE}} & \multicolumn{1}{l}{\textbf{MAE}} \\
    \hline
    SVM & 4.14 & 0.47 & 2.82  \\
    Mean & 6.20 & 0.71 & 4.55 \\
    Random Guess & 18.23 & 2.11 & 14.93\\
    RNN & 4.29 & 0.49 & 2.96 \\ 
    GRU & 4.48 & 0.51 & 3.11  \\ 
    LSTM Mult & 4.99 &  0.57 & 3.58  \\ 
    LSTM Uni & 4.74 &  0.54 & 3.37  \\ 
    Random Forest & 4.17 & 0.48 & 2.91 \\
    \hline
    \end{tabular}
    \end{center}
\end{table}


Como a normalização não trouxe muitas melhorias aos modelos, decidimos por continuar os testes utilizando os valores não normalizados. O próximo passo foi decidir os melhores valores para cada parâmetro e hiper-parâmetro. Todos os modelos tiveram uma melhora após os testes com o hyperas, abaixo podem ser vistas os 3 modelos com melhores evoluções utilizando a ferramenta.


\begin{table}[H]
    \caption{Predição do modelo X1 com e sem melhores parâmentros}
    \label{table:RmseComparison}
    \begin{center}
    \begin{tabular}{ccccccc}
    \hline
    \multicolumn{1}{l}{\textbf{Modelo}} & \multicolumn{1}{l}{\textbf{RMSE}} & \multicolumn{1}{l}{\textbf{NRMSE}} & \multicolumn{1}{l}{\textbf{MAE}} \\
    \hline
    SVM & 4.14 & 0.47 & 2.82  \\
    Mean & 6.20 & 0.71 & 4.55 \\
    Random Guess & 18.23 & 2.11 & 14.93\\
    RNN & 4.29 & 0.49 & 2.96 \\ 
    GRU & 4.48 & 0.51 & 3.11  \\ 
    LSTM Mult & 4.99 &  0.57 & 3.58  \\ 
    LSTM Uni & 4.74 &  0.54 & 3.37  \\ 
    Random Forest & 4.17 & 0.48 & 2.91 \\
    \hline
    \end{tabular}
    \end{center}
\end{table}

\begin{table}[H]
    \caption{Predição do Modelo X2 com e sem os melhores parâmetros}
    \label{table:RmseComparison}
    \begin{center}
    \begin{tabular}{ccccccc}
    \hline
    \multicolumn{1}{l}{\textbf{Modelo}} & \multicolumn{1}{l}{\textbf{RMSE}} & \multicolumn{1}{l}{\textbf{NRMSE}} & \multicolumn{1}{l}{\textbf{MAE}} \\
    \hline
    SVM & 4.14 & 0.47 & 2.82  \\
    Mean & 6.20 & 0.71 & 4.55 \\
    Random Guess & 18.23 & 2.11 & 14.93\\
    RNN & 4.29 & 0.49 & 2.96 \\ 
    GRU & 4.48 & 0.51 & 3.11  \\ 
    LSTM Mult & 4.99 &  0.57 & 3.58  \\ 
    LSTM Uni & 4.74 &  0.54 & 3.37  \\ 
    Random Forest & 4.17 & 0.48 & 2.91 \\
    \hline
    \end{tabular}
    \end{center}
\end{table}]


\begin{table}[H]
    \caption{Predição do modelo X3 com e sem seus melhores parâmetros}
    \label{table:RmseComparison}
    \begin{center}
    \begin{tabular}{ccccccc}
    \hline
    \multicolumn{1}{l}{\textbf{Modelo}} & \multicolumn{1}{l}{\textbf{RMSE}} & \multicolumn{1}{l}{\textbf{NRMSE}} & \multicolumn{1}{l}{\textbf{MAE}} \\
    \hline
    SVM & 4.14 & 0.47 & 2.82  \\
    Mean & 6.20 & 0.71 & 4.55 \\
    Random Guess & 18.23 & 2.11 & 14.93\\
    RNN & 4.29 & 0.49 & 2.96 \\ 
    GRU & 4.48 & 0.51 & 3.11  \\ 
    LSTM Mult & 4.99 &  0.57 & 3.58  \\ 
    LSTM Uni & 4.74 &  0.54 & 3.37  \\ 
    Random Forest & 4.17 & 0.48 & 2.91 \\
    \hline
    \end{tabular}
    \end{center}
\end{table}

\begin{table}[H]
    \caption{Predição do modelo X3 com e sem seus melhores parâmetros}
    \label{table:RmseComparison}
    \begin{center}
    \begin{tabular}{ccccccc}
    \hline
    \multicolumn{1}{l}{\textbf{Modelo}} & \multicolumn{1}{l}{\textbf{RMSE}} & \multicolumn{1}{l}{\textbf{NRMSE}} & \multicolumn{1}{l}{\textbf{MAE}} \\
    \hline
    SVM & 4.14 & 0.47 & 2.82  \\
    Mean & 6.20 & 0.71 & 4.55 \\
    Random Guess & 18.23 & 2.11 & 14.93\\
    RNN & 4.29 & 0.49 & 2.96 \\ 
    GRU & 4.48 & 0.51 & 3.11  \\ 
    LSTM Mult & 4.99 &  0.57 & 3.58  \\ 
    LSTM Uni & 4.74 &  0.54 & 3.37  \\ 
    Random Forest & 4.17 & 0.48 & 2.91 \\
    \hline
    \end{tabular}
    \end{center}
\end{table}

Com todos os modelos nas suas melhores versões, com os valores mais adequados de parâmetros e hiper-parâmetros, realizamos um teste comparando todos os modelos com o mesmo conjunto de dados e para os seguintes valores de Janela: 

\begin{table}[H]
    \caption{Resultados com a Janela 1}
    \label{table:RmseComparison}
    \begin{center}
    \begin{tabular}{ccccccc}
    \hline
    \multicolumn{1}{l}{\textbf{Modelo}} & \multicolumn{1}{l}{\textbf{RMSE}} & \multicolumn{1}{l}{\textbf{NRMSE}} & \multicolumn{1}{l}{\textbf{MAE}} \\
    \hline
    SVM & 4.14 & 0.47 & 2.82  \\
    Mean & 6.20 & 0.71 & 4.55 \\
    Random Guess & 18.23 & 2.11 & 14.93\\
    RNN & 4.29 & 0.49 & 2.96 \\ 
    GRU & 4.48 & 0.51 & 3.11  \\ 
    LSTM Mult & 4.99 &  0.57 & 3.58  \\ 
    LSTM Uni & 4.74 &  0.54 & 3.37  \\ 
    Random Forest & 4.17 & 0.48 & 2.91 \\
    \hline
    \end{tabular}
    \end{center}
\end{table}



\begin{table}[H]
    \caption{Resultados com a janela 2}
    \label{table:RmseComparison}
    \begin{center}
    \begin{tabular}{ccccccc}
    \hline
    \multicolumn{1}{l}{\textbf{Modelo}} & \multicolumn{1}{l}{\textbf{RMSE}} & \multicolumn{1}{l}{\textbf{NRMSE}} & \multicolumn{1}{l}{\textbf{MAE}} \\
    \hline
    SVM & 4.14 & 0.47 & 2.82  \\
    Mean & 6.20 & 0.71 & 4.55 \\
    Random Guess & 18.23 & 2.11 & 14.93\\
    RNN & 4.29 & 0.49 & 2.96 \\ 
    GRU & 4.48 & 0.51 & 3.11  \\ 
    LSTM Mult & 4.99 &  0.57 & 3.58  \\ 
    LSTM Uni & 4.74 &  0.54 & 3.37  \\ 
    Random Forest & 4.17 & 0.48 & 2.91 \\
    \hline
    \end{tabular}
    \end{center}
\end{table}

\begin{table}[H]
    \caption{Resultados com a janela 3}
    \label{table:RmseComparison}
    \begin{center}
    \begin{tabular}{ccccccc}
    \hline
    \multicolumn{1}{l}{\textbf{Modelo}} & \multicolumn{1}{l}{\textbf{RMSE}} & \multicolumn{1}{l}{\textbf{NRMSE}} & \multicolumn{1}{l}{\textbf{MAE}} \\
    \hline
    SVM & 4.14 & 0.47 & 2.82  \\
    Mean & 6.20 & 0.71 & 4.55 \\
    Random Guess & 18.23 & 2.11 & 14.93\\
    RNN & 4.29 & 0.49 & 2.96 \\ 
    GRU & 4.48 & 0.51 & 3.11  \\ 
    LSTM Mult & 4.99 &  0.57 & 3.58  \\ 
    LSTM Uni & 4.74 &  0.54 & 3.37  \\ 
    Random Forest & 4.17 & 0.48 & 2.91 \\
    \hline
    \end{tabular}
    \end{center}
\end{table}

Como pode ser visto na tabela X,  todos  os  modelos  de aprendizagem  profunda  (LSTM  uni-variado  e  multi-variado,RNN  e  GRU)  tiveram  resultados  semelhantes,  mas  não  tão bons quanto os de aprendizagem supervisionada comum, como o random forest. Isso  pode  ter  acontecido  devido  ao  nosso  conjunto  de dados  e  seu  tamanho.  Redes  neurais  recorrentes  precisam de  um  grande  volume  de  dados  para  mapear  e  aprender  a sua  distribuição,  ao  contrário  de  modelos  de  aprendizagem supervisionada  tradicionais, que obtiveram melhores previsões com o nosso dataset.

Outro  ponto  interessante  a  se  notar ao observar a tabela y  é  o  tempo  de  treinamento de cada método. Os modelos de aprendizagem profunda tiveram um tempo de treinamento consideravelmente maior se comparado aos demais, o que é esperado, visto que possuem muito mais camadas de processamento. Já os  modelos  utilizados  como  base  de  comparação  tiveram um  tempo  de treinamento  extremamente  rápido,  pois  são métodos triviais  e  não  exigem  muito  processamento  e,  por consequência, também tiveram as piores previsões.


\begin{table}[H]
    \caption{Tabela Y com tempo de treinamento de cada modelo}
    \label{table:RmseComparison}
    \begin{center}
    \begin{tabular}{ccccccc}
    \hline
    \multicolumn{1}{l}{\textbf{Modelo}} & \multicolumn{1}{l}{\textbf{RMSE}} & \multicolumn{1}{l}{\textbf{NRMSE}} & \multicolumn{1}{l}{\textbf{MAE}} \\
    \hline
    SVM & 4.14 & 0.47 & 2.82  \\
    Mean & 6.20 & 0.71 & 4.55 \\
    Random Guess & 18.23 & 2.11 & 14.93\\
    RNN & 4.29 & 0.49 & 2.96 \\ 
    GRU & 4.48 & 0.51 & 3.11  \\ 
    LSTM Mult & 4.99 &  0.57 & 3.58  \\ 
    LSTM Uni & 4.74 &  0.54 & 3.37  \\ 
    Random Forest & 4.17 & 0.48 & 2.91 \\
    \hline
    \end{tabular}
    \end{center}
\end{table}