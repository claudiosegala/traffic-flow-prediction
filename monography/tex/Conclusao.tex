Essa monografia apresentou um trabalho de comparação de modelos tradicionais de aprendizagem de máquina com modelos de aprendizagem profunda para predição de fluxo de veículos. Por meio dos resultados das comparações, foi possível confirmar a hipótese de que modelos utilizados para predição de fluxo em vias livres podem ser aplicados a cenários de vias com cruzamentos.

Primeiramente, foi discutido o tráfego e suas características, assim como os fatores que podem influenciar no seu comportamento. Foi visto que existem os fatores espaciais, temporais e aleatórios, assim como os motivos que foram considerados para não utilizar-se os fatores aleatórios. Subsequentemente, foi feita uma análise dos dados utilizados como entrada dos modelos de predição. Foram explicitadas todas as colunas disponíveis nos registros disponibilizados pelo \textit{\acrshort{DETRAN}}, assim como quais destas colunas foram descartadas e porquê. 

Discutiu-se também a metodologia empregada nos experimentos. Mostrou-se como foi feita a transformação dos dados do conjunto inicial de registros para um outro conjunto contendo o fluxo de veículos por intervalo de tempo. Também foi explicado o tratamento adicional dado a algumas colunas, como a técnica de \textit{One Hot Enconding} que foi utilizada na coluna \textit{Data}. Por último, nesta seção também foi feita uma análise de sazonalidade dos dados, o que explicou o motivo de certos períodos de tempo terem apresentado valores menores de fluxo.

Ainda referente a metodologia, também foi explicado quais os modelos utilizados para as comparações e experimentos, assim como, de que maneira eles diferem entre si e quais as caraterísticas principais de cada um. Por último, foi mostrado como se deu a escolha dos melhores valores de parâmetros e hiper-parâmetros.

Por fim, foram mostrados os resultados dos experimentos para a predição de fluxo de veículos para 15, 30, 45 e 60 minutos no futuro. Como esperado, todos os modelos de aprendizagem de máquina tiveram melhores resultados que os modelos de base de comparação. Também foi possível observar que dentre os modelos propostos, o que apresentou os melhores valores de predição, no geral, foi o \textit{\acrshort{RF}}. Portanto, todos os objetivos e resultados esperados traçados no início desta monografia foram atingidos.


\section{Trabalhos Futuros}

Este trabalho possui espaço para melhorias e ajustes que fugiram do escopo inicialmente definido. Dentre os trabalhos futuros possíveis, seguem alguns dos mais relevantes:

\begin{itemize}
    \item Treinar os modelos com pelo 1 ano de dados;
    
    Isso permitiria uma abordagem mais abrangente quanto a sazonalidade dos dados e como eles impactam no comportamento do fluxo. Além disso, os dados agora teriam incorporados as alterações que eventos ao longo do ano causam no fluxo.
    
    \item Polir a representação da coluna  \textit{Data} para os modelos;
    
    Simplificamos a coluna \textit{Data} para uma representação referente somente ao dia da semana, porém, com uma representação mais complexa utilizando também a hora do dia, os modelos talvez se mostrassem mais eficazes.
    %TODO: Verificar esse item, pois as metricas (RMSE, MAE, etc) ja servem para essa avaliacao e, talvez, colocar este item desqualifique um dos obejtivos do trabalho
    \item Aplicar os modelos desenvolvidos a um conjunto de dados referente a vias livres;
    
    Ao se ter o desempenho dos modelos em vias livres, seria possível ter uma noção de como os modelos treinados para cruzamento se saem em relação a modelos treinados em vias livres.
    
    \item Utilizar múltiplos sensores para predição;
    
    Nesse trabalho utilizou-se apenas um dos sensores disponíveis, porém as informações dos outros sensores que estão próximos poderiam auxiliar na previsão.
    
    \item Utilizar modelos paramétricos;
    
    Devido a limitações de hardware e uma performance inicial ruim, decidiu-se retirar o \textit{\acrfull{ARIMA}} e o \textit{\acrfull{LR}}. Porém, existem outros modelos paramétricos que podem ser mais próprios para conjunto de dados utilizado como \textit{\acrfull{SARIMA}} que, talvez, tenha uma performance melhor.
    
    \item Incluir modelos não paramétricos mais complexos;
    
    Devido a limitações de hardware e tempo, o trabalho focou em arquiteturas simples em vez de arquiteturas como \textit{\acrshort{SLSTM}} discutidos no capítulo \ref{chapter:trabalhos_relacionados}. Estes poderiam ter uma performance melhor que os modelos mais simples.
    
    \item Aprofundar a escolha de parâmetros e hiper-parâmetros com mais opções;
    
    Nem todos os parâmetros e hiper-parâmetros de cada modelo foram explorado nesse trabalho. Sendo assim, é possível que possa haver uma melhora de performance dos modelos atuais.
    
    \item Incluir modelos diferentes;
    
    Não utilizou-se nenhum modelo \textit{boosted} nos experimentos, podendo ser que esse, ou outros tipos consigam traçar melhor a distribuição dos dados.
    
\end{itemize}

% TODO: Dizer a proposta do TCC. Resumir as etapas das metodologias propostas para alcançar o objetivo do método proposto
    % Descrever os resultados encontrados
    % Descrever as vantagens, desvantagens e dificuldades encontradas
    % Enaltecer as contribuições do trabalho