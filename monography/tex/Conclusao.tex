% TODO?: Adicionar Ensemble Methods (7 LSTM, uma para cada dia da semana)

\section{Considerações Finais}

Este trabalho compara 5 modelos diferentes em suas formas 

\section{Trabalhos Futuros}

Como demonstrado, este trabalho possui limitações e ainda tem muito espaço para explorar e melhorar. Sendo eles:

\begin{itemize}
    \item Treinar os modelos com pelo 1 ano de dados;
    \item Verificar como os modelos reagem com uma representação mais complexa do dia e hora;
    \item Aplicar os modelos desenvolvidos com um conjunto de dados de vias urbanas expressas para comparação;
    \item Verificar se utilizarmos múltiplos sensores haveria uma melhora das métricas;
    \item Utilizar modelos paramétricos para servir de comparação, como \textit{\acrshort{ARIMA}}.
    \item Utilizar modelos mais complexos como \textit{\acrshort{SLSTM}} discutidos no capítulo \ref{chapter:trabalhos_relacionados};
    \item Utilizar um intervalo maior de dados para verificar se os modelos poderiam melhorar com um espaço de treino maior;
    \item Aprofundar a escolha de parâmetros e hiper-parâmetros com mais opções.
    % TODO: cite articles that prove this
    \item Utilizar versão \textit{boosted} da \textit{\acrshort{RF}} que tem se mostrado superior em problemas onde a versão \textit{bootstrapping} se saiu bem.
\end{itemize}