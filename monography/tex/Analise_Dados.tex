% TODO: Usar a análise de dados do Fábio como inspiração para ver se tem mais algo que podemos escrever aqui

Neste capítulo, é apresentada uma análise do conjunto de dados utilizado para treinar os modelos para realizar suas predições. Primeiramente, é explicado de que forma o tráfego e suas características podem influenciar nos registros. Subsequentemente, é explicitado como os dados foram adquiridos e suas características.

\section{Características do Tráfego}

% TODO: arranjar referência
É importante salientar que o tráfego é afetado por diversos fatores: temporais, espaciais e aleatórios. Os fatores temporais são, por exemplo, o horário do expediente dos trabalhadores que possuem veículos. Os fatores espaciais são, por exemplo, como a malha rodoviário se distribui em uma localização, ou a quantidade de empresas em uma área. Já fatores aleatórios são, por exemplo, acidentes que possam vir a ocorrer, ou meteorologia do dia. Sendo assim, modelos de previsões podem ter desempenhos diferentes dependendo de onde e quando os dados forem coletados. Isso sem levar em conta a impossibilidade de se prever os fatores aleatórios.

% TODO: botar referência
Dessa forma, é comum na literatura optar-se por usar apenas dados com fatores espaciais e temporais. Ao desconsiderar-se o fator aleatório, é feita uma suposição de que exista uma tendência no fluxo que pode ser prevista e que, embora as variáveis aleatórias possam afetar o fluxo, elas não afetam o bastante para mudar a tendência no longo prazo. Portanto, seguindo a literatura, esse artigo utilizará variáveis temporais e espaciais.

\section{Aquisição}

Os dados\footnote{http://bit.ly/processed-data-2l5MaAG} provêm de dois cruzamentos na avenida Hélio Prates que pode ser vista na Figura \ref{figure:helio}. Esta cruza Taguatinga e Ceilândia, que são duas cidades satélites do \acrfull{DF}. Estes dados foram coletados por fiscalizadores eletrônicos localizados nos cruzamentos e foram fornecidos pelo \acrfull{DETRAN}\footnote{http://www.detran.df.gov.br/} do \acrshort{DF}. Neste conjunto de dados estão inclusos registros de todos os veículos que passaram pelo local nos meses de maio a julho de 2016 em forma de uma série temporal. Mais especificamente, do primeiro dia de maio ao último de julho de 2016, totalizando 13 semanas e 1 dia, 92 dias. 

\begin{table}[h]
    \begin{tabular}{ccccccc}
    \toprule
    \multicolumn{1}{l}{\textbf{Id Equipamento}} & \multicolumn{1}{l}{\textbf{Data}} & \multicolumn{1}{l}{\textbf{Hora}} & \multicolumn{1}{l}{\textbf{Faixa}} & \multicolumn{1}{l}{\textbf{km/h}} & \multicolumn{1}{l}{\textbf{km/h Max}} & \multicolumn{1}{l}{\textbf{Tamanho}} \\ 
    \midrule
        RSI128 & 2016/05/01 & 00:00:09 & 1 & 20 & 60 & 0.0 \\
    \midrule
    RSI131 & 2016/05/01 & 00:00:09 & 2 & 45 & 60 & 1.1 \\
    \midrule
    RSI132 & 2016/05/01 & 00:00:09 & 1 & 40 & 60 & 0.0 \\
    \midrule
    RSI131 & 2016/05/01 & 00:00:10 & 1 & 35 & 60 & 0.5 \\
    \midrule 
    RSI129 & 2016/05/01 & 00:00:12 & 1 & 35 & 60 & 0.0 \\
    \midrule
    RSI131 & 2016/05/01 & 00:00:13 & 1 & 43 & 60 & 1.0 \\
    \midrule
    RSI131 & 2016/05/01 & 00:00:14 & 1 & 35 & 60 & 1.2 \\
    \midrule
    RSI128 & 2016/05/01 & 00:00:18 & 1 & 32 & 60 & 0.0 \\
    \midrule
    RSI131 & 2016/05/01 & 00:00:19 & 2 & 41 & 60 & 1.1 \\
    \midrule
    RSI129 & 2016/05/01 & 00:00:20 & 1 & 37 & 60 & 0.0 \\
    \midrule
    RSI131 & 2016/05/01 & 00:00:22 & 1 & 49 & 60 & 1.5 \\
    \midrule
    RSI018 & 2016/05/01 & 00:00:36 & 2 & 47 & 60 & 0.0 \\
    \midrule
    RSI018 & 2016/05/01 & 00:00:43 & 2 & 40 & 60 & 0.0 \\
    \midrule
    RSI018 & 2016/05/01 & 00:00:45 & 2 & 37 & 60 & 0.0 \\
    \midrule
    RSI033 & 2016/05/01 & 00:00:52 & 2 & 17 & 60 & 0.0 \\
    \bottomrule
    \end{tabular}
    \label{table:data}
    \caption{Exemplo dos dados recebidos coletados pelo \acrshort{DETRAN}}
\end{table}

% TODO: verificar com Vinícius se isso ta correto
Quanto as características das colunas, o conjunto de dados possui tanto colunas quantitativas quanto qualitativas. Só existe uma coluna qualitativa, sendo ela ordinal, pois em seu nome ocorre uma numeração. Todas as outras colunas são quantitativas, sendo a de tamanho a única contínua e o resto discreto. Porém a coluna de km/h Max tem seu valor constante constante. 

Para cada registro se tem uma identificação do fiscalizador eletrônico, data, hora, faixa de via, velocidade, limite de velocidade da via e tamanho do veículo, assim como mostrado na tabela \ref{table:data}. Os registros são de 8 sensores com um total de 10.801.781 registros de veículos, sendo a quantidade de registros por sensor não distribuídos igualmente:

\begin{itemize}
    \item \textbf{RSI033:} 2.382.754 registros de veículos
    \item \textbf{RSI032:} 2.117.820 registros de veículos
    \item \textbf{RSI018:} 2.029.559 registros de veículos
    \item \textbf{RSI017:} 1.686.900 registros de veículos
    \item \textbf{RSI131:} 816.219 registros de veículos
    \item \textbf{RSI132:} 652.998 registros de veículos
    \item \textbf{RSI129:} 578.652 registros de veículos
    \item \textbf{RSI128:} 536.879 registros de veículos
\end{itemize}

Vale salientar que não foram encontrados atributos incompletos ou nulos. Porém uma das colunas possui dados inconsistentes com a realidade. Isso acontece pois, como pode ser visto na Tabela \ref{table:data} existem veículos com tamanho 0, o que é impossível. Por último, mas não menos importante, também deve-se notar que os fiscalizadores eletrônicos utilizados ficam próximos de semáforos, ou seja, existem momentos em que o fluxo diminui bastante, devido aos sinais vermelhos. 

\begin{figure}[t]
    \centering
    \includegraphics[scale=0.45,angle=90]{monography/img/avenida_helio_prates.png}
    \label{figure:helio}
    \caption{Visão Espacial da Avenida Hélio Prates Provindos do Apple Maps}
\end{figure}