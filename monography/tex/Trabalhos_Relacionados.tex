% TODO: adicionar explicação de como vai funcionar o capitulo, quais artigos võa ser detalhados e um ultra resumo do que eles consistem.

% TODO: Deve ter pelo menos 5 artigos explicados nesse capitulo

% TODO: Adicionar critica a todos eles quando achar. Para o urbano falar que ele utilizou o dados de diferentes tipos de vias que podem apresentar comportamentos diferentes.

\label{chapter:trabalhos_relacionados}

O objetivo deste capítulo é apresentar trabalhos relacionados à predição de tráfego urbano, mostrar suas contribuições e contextualizar o trabalho atual. O primeiro trabalho mencionado é sobre predição de fluxo utilizando \textit{\acrshort{LSTM}}. O segundo também utiliza aprendizagem de máquina, explorando o modelo \textit{\acrshort{SAE}} para prever o fluxo de veículos em vias da Califórnia. Por último, é apresentado um trabalho que utiliza \acrshort{GRU}.

\section{Aprendizagem Profunda}

No trabalho de Zainab Abbas et. al. em \textit{Short-term traffic prediction using long short-term memory neural networks} \cite{Zainab_2018} é apresentado um trabalho de predição de tráfego (calculando a densidade da via em vez do fluxo, como é feito usualmente) que tem como objetivo antecipar congestionamentos em rodovias de Estocolmo, Suécia, utilizando aprendizagem profunda. No caso, o autor propõe uma variação do \textit{\acrshort{LSTM}}, mais precisamente \textit{\acrfull{SLSTM}}. Para tal, são utilizados dados coletados por sensores do sistema de controle da autoestrada da cidade. Tais sensores monitoram as principais vias da metrópole e coletam informações como fluxo e velocidade de cada faixa a cada 1 minuto. O trabalho propõe três modelos de predição de densidade da via:
 
 % TODO: colocar a imagem que o artigo utiliza para ficar mais fácil entender
\begin{itemize}
    \item O (1-1) Modelo utilizando apenas um sensor que faz a predição apenas do local daquele sensor
    \item O (n-n) Modelo que utiliza n sensores de uma determinada área e faz a predição de todas as localidades
    \item O (m-n) Modelo que utiliza os m sensores mais significantes de uma área que contém um total de n sensores e que faz a predição para todos os n locais.
\end{itemize}

\begin{equation}\label{eq:solve}
k = q / v
\end{equation}
Fórmula do cálculo da densidade utilizada pelo trabalho ~\ref{eq:solve}

Dos três modelos apresentados, o mais eficiente foi o m-n, pois faz a previsão de tráfego em vários pontos diferentes da via e tem um custo computacional menor que o n-n, além disso, utiliza menos sensores, o que diminui os dados de entrada da rede neural. Esse Modelo foi comparado com outros usando as métricas \textit{\acrshort{RMSE}} e \textit{\acrshort{MAE}}. O resultado dessas métricas é comparado com a acurácia de outros modelos como \textit{\acrfull{RNN}} e \textit{\acrfull{FFN}}. O erro calculado foi menor na metodologia proposta pelo autor em todos os casos, comprovando a eficácia do modelo apresentado (SLSTM).

No artigo \textit{Traffic Flow Prediction with Big Data: A Deep Learning Approach} \cite{lv_6894591} por Yisheng Lv et. al. foi proposto um trabalho semelhante de comparação de modelos de predição, mas utilizando \textit{\acrfull{SAE}}. O modelo de predição proposto pelo artigo é aplicado aos dados coletados por 15,000 sensores espalhados pelas estradas da Califórnia. As predições são divididas em intervalos de 15, 30, 45 e 60 minutos. Mais uma vez, a acurácia dos testes foi medida utilizando  \acrshort{MAE}, \acrshort{MRE} e \acrshort{RMSE} para cada intervalo de tempo e comparada com a acurácia de predição de outros métodos, como \acrfull{BPNN}, \acrfull{RW}, \acrfull{SVM} e \acrfull{RBFNN}. Novamente, o método apresentado pelo artigo foi o mais eficiente.

Um modelo mais próximo do utilizado em nossos testes é descrito em \textit{Long Short-Term Memory Recurrent Neural Network For Urban Traffic Prediction: A Case Study of Seoul} \cite{Seoul} por Yong-Ju Lee et. al.. Diferentemente dos trabalhos relacionados apresentados nos parágrafos anteriores, esse artigo apresenta um estudo feito com dados coletados de vias urbanas de Seul, e não somente de rodovias expressas. O trabalha utiliza 3 base de dados diferentes:

\begin{itemize}
    \item Uma base gerada sintéticamente pela \textit{SK Planet Company}.
    \item Uma base coletada do governo metropolitano de Seul.
    \item Uma base de dados coletado do \textit{Aplicatio T-Map}, considerado a aplicação de navegação mais utilizada da Coreia.
\end{itemize} 

Além das 3 bases de dados, também foi adicionado ao modelo uma variável para medir o impacto das condições climáticas sobre o tráfego.
Para a predição, foram utilizados o \textit{\acrshort{LSTM}}, \textit{\acrshort{GRU}} e um método proposto chamado \textit{\acrfull{MLSTM}}. Nos testes feitos, o método de avaliação foi o \textit{\acrshort{RMSE}}, para o qual o \textit{\acrshort{MLSTM}} se mostrou mais eficiente, seguido do \textit{\acrshort{LSTM}} comum e do \textit{\acrshort{GRU}}. Porém, a diferença do RMSE entre os 3 algoritmos não ultrapassou 0.1, o que demonstra que a eficácia de todos os métodos é bem similar para o conjunto de dados do trabalho. Vale ressaltar também que todos os modelos utilizados neste trabalho se encaixam na categoria de aprendizagem profunda e derivam de redes neurais recorrentes, o que pode explicar a performance tão similar.

Em \textit{Using LSTM and GRU Neural Network Methods for Traffic Flow Prediction} \cite{fu2016using} por Rui Fu et. al. tem-se novamente o mesmo problema de predição de tráfego. Porém, segundo o autor, é apresentado pela primeira vez na literatura uma proposta de utilização de \textit{\acrfull{GRU}} para esse tipo de problema. No caso, o autor compara três modelos, sendo eles \textit{\acrshort{ARIMA}}, \textit{\acrshort{LSTM}} e \textit{\acrshort{GRU}}. Os dados utilizados pelo trabalho citado foram disponibilizados publicamente pelo governo, \textit{PeMS}, mais especificamente de \textit{Bay Area}, \textit{Alameda}, \textit{Oakland} nos Estados Unidos. 

Para o treinamento foram utilizados 50 séries temporais, cada um gerada por um sensor diferente. Já para as métricas similares aos outros artigos, \textit{\acrshort{MSE}} e \textit{\acrshort{MAE}}. No final do experimento os modelos tinham suas métricas similares, sem um diferença muito significativa. Porém com o modelo \textit{\acrshort{GRU}} sendo superior a \textit{\acrshort{LSTM}} por 84\% da série temporal e os modelos não-paramêtricos demonstrando uma melhor capacidade de se ajustar aos dados.
% TODO: adicionar critica de que ele poderia ter feito uma seleção de sensores por porximidade para tentar melhorar os resultados

\section{Algoritmos Tradicionais}

Yanyan Xu et. a. em \textit{Short-term Traffic Volume Prediction using Classification and Regression Trees} \cite{xu2013short} ...

Yi Hou et al. em \textit{Traffic Flow Forecasting for Urban Work Zones} \cite{hou2014traffic} ...
