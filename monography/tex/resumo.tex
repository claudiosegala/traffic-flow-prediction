Predição de fluxo de veículos é uma tarefa complexa. Existem diversos fatores desafiadores como a oscilação do fluxo e fatores externos como acidentes e o clima. A literatura relacionada foca em predições de fluxo em vias livres, sem obstruções, esta monografia tem como base cruzamentos, que possuem semáforos e obstruções. Ou seja, esta monografia tem como objetivo verificar se os modelos utilizados em vias livres conseguem ser aplicados com sucesso em cruzamentos.  Para realizar a predição de fluxo nesse cenário, foram testadas quatro técnicas de aprendizado de máquina. Dentre elas, duas utilizam aprendizagem profunda, \textit{\acrfull{LSTM}} e \textit{\acrfull{GRU}}, e duas são tradicionais, \textit{\acrfull{SVM}} e \textit{\acrfull{RF}}. Os resultados dos experimento foram promissores, visto que todas as técnicas apresentaram um desempenho superior às bases de comparações propostas. Também foi mostrado que os modelos de aprendizado de máquina tradicionais se destacaram em relação aos de aprendizagem profunda, com o modelo \acrshort{SVM} obtendo os melhores resultados para todas as predições (de 15, 30, 45 e 60 minutos no futuro).