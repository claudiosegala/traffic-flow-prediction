Predição de fluxo de veículos é uma tarefa complexa. Existem diversos fatores desafiadores que devem ser levados em consideração, como a oscilação do fluxo, acidentes automobilísticos e o clima local (fatores externos). A literatura relacionada tende a se concentrar em predições de fluxo em vias livres, sem obstruções. Esta monografia tem como foco vias com cruzamentos, que possuem semáforos e obstruções. Ou seja, este trabalho tem como objetivo verificar se os modelos utilizados em vias livres podem ser aplicados com sucesso a vias com cruzamentos.  Para realizar a predição de fluxo nesse cenário, foram testadas quatro técnicas de aprendizado de máquina. Dentre elas, duas utilizam aprendizado profundo, \textit{\acrfull{LSTM}} e \textit{\acrfull{GRU}}, e duas são tradicionais, \textit{\acrfull{SVM}} e \textit{\acrfull{RF}}. Os resultados dos experimento foram promissores, visto que todas as técnicas apresentaram um desempenho superior às bases de comparações propostas. Também foi mostrado que os modelos de aprendizado de máquina tradicionais se destacaram em relação aos de aprendizado profundo, com o modelo \acrshort{SVM} obtendo os melhores resultados para todas as predições (de 15, 30, 45 e 60 minutos no futuro).

