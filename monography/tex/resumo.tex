Predição de fluxo de veículos é uma tarefa complexa. Ela possui diversos fatores desafiadores como a escalabilidade, a oscilação do fluxo e fatores exógenos. A literatura foca em vias livres, sem obstruções. Já esta monografia foca em cruzamentos, que possuem obstruções. Para resolver o problema, foram testadas quatro técnicas de aprendizagem de máquina. Sendo que todas elas já foram utilizadas para o problema de vias livres. Duas das técnicas são de aprendizagem profunda, \acrfull{LSTM} e \textit{\acrfull{GRU}}, e duas são tradicionais, \textit{\acrfull{SVM}} e \textit{\acrfull{RF}}. Os resultados do experimento foram promisserem. Todas as técnicas tiveram um desempenho superior as bases de comparações propostas. Além disso, as técnicas tradicionais se destacaram em relação as de aprendizagem profunda testadas. No quesito precisão as técnicas testadas conseguiram prever se o fluxo ia diminuir ou aumentar 70\% das vezes.