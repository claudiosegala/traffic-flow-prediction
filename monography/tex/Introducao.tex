A frota  de  veículos no Brasil tem aumentado ao longo  dos últimos anos. De acordo com a \acrfull{ANS}, em 2018, o Brasil já apresentava 1 carro para cada 4 habitantes. Isso equivale, aproximadamente, a 50 milhões de automóveis em circulação \cite{G1}. As vias públicas não tem conseguido acompanhar este crescimento e acomodar todo este volume. Brasília, por exemplo, que é uma cidade planejada, já começa a sentir os efeitos causados por uma grande frota de veículos. Como pode ser visto no Portal Brasileiro de Dados Abertos \cite{detran_2018}, a capital do país já se vê obrigada a comportar 3.5 milhões de automóveis e, por conta disso, tem sofrido com congestionamentos. Brasília não é uma exceção, todos os grandes centros urbanos do país estão sendo afetados pelas mazelas do crescimento do setor automobilístico, e pensar em soluções para estes problemas é mais que necessário.

 Em resposta a essa tendência, as metrópoles tem utilizado da tecnologia para gerenciar melhor o trânsito em suas vias, ao passo que os motoristas têm feito uso cada vez mais intenso de aplicativos para escapar de congestionamentos e descobrir os caminhos mais rápidos até seus destinos. Porém, se essas tecnologias e aplicações não se preocuparem em como o fluxo da via estará nos próximos minutos, erros de cálculo podem ser cometidos.

Mas se não há como prever o futuro, como planejar-se melhor? Utilizando-se de técnicas de aprendizagem de máquina é possível aprender como uma via tende a se comportar, tornando possível estimar como o seu fluxo de veículos estará em alguns minutos.

\section{Motivação}

Os grandes centros urbanos não conseguem acompanhar a constante expansão do volume de carros. Por conta disso, congestionamentos tendem a se tornar mais frequentes e mais severos, aumentando a emissão de gases poluentes na atmosfera, contribuindo para a poluição sonora e aumentando o tempo gasto pelos motoristas no trânsito, impactando de maneira negativa na qualidade de vida da cidade e seus habitantes.

Neste contexto, surgiram várias pesquisas na área de mobilidade urbana e transporte que visam contornar estes problemas. Dentre essas pesquisas, um dos meios encontrados de amenizar o congestionamento é tentar prever quando ele irá acontecer. Uma vez que os motoristas saibam que um engarrafamento irá ocorrer, eles podem se planejar melhor e navegar por rotas alternativas, distribuindo de maneira mais eficiente os carros na malha viária. Já organizações governamentais poderiam utilizar os dados para obter uma noção em tempo real de como estará o fluxo pela cidade, possibilitando o redirecionamento das vias e reprogramação de semáforos. Além disso, a previsão do tráfego possibilitaria a implementação ou a evolução de sistemas mais complexos, como um sistema de controle semafórico dinâmico ou sistema de auxílio de decisão para controle da malha viária.

A literatura está repleta de artigos que tentam prever a quantidade de veículos que irão passar em uma via em um determinado intervalo de tempo (predição de fluxo) utilizando aprendizagem de máquina \cite{doi:10.1080/01441647.2014.992496, fu2016using, hamed_prediction_1995, lv_6894591, Seoul, wang_2018, Xiaolei_2015, Zainab_2018}. Porém, a maioria desses artigos coletam dados e fazem suas análises em rodovias expressas, enquanto alguns trabalham com outros tipos de rodovia, mas de forma mais genérica \cite{Seoul}. Nessas vias não há intersecções, ou barreiras semafóricas. Nosso estudo tem como desafio realizar a predição de fluxo em cenários que se assemelhem mais a centros urbanos, onde as vias possuem semáforos e cruzamentos que podem impactar de maneira significativa no comportamento do fluxo de veículos.

\section{Hipótese}

Tendo em mente todas as informações citadas anteriormente, este trabalho tem como hipótese que os modelos utilizados para predição de fluxo de veículos em vias livres serão capazes de prever, também a curto prazo, o fluxo de veículos em cruzamentos. Mais especificamente, serão capazes de prever 15, 30, 45 e 60 minutos no futuro. Sobretudo, eles serão capazes também de prever o fluxo mais precisamente do que as bases de comparação, os modelos \textit{Moving Average} e \textit{Naive}.

\section{Objetivo}

Este trabalho busca realizar a predição de fluxo de veículos em vias urbanas a curto prazo, mais especificamente, para 15, 30, 45 e 60 minutos no futuro. Para tal fim, serão utilizados os modelos \textit{\acrfull{LSTM}}, \textit{\acrfull{GRU}}, \textit{\acrfull{RF}} e \textit{\acrfull{SVM}}. Como base de comparação, serão utilizados os modelos \textit{Moving Average} e \textit{Naive}. Portanto, este trabalho tem como objetivo propor uma análise do comportamento destes mesmos métodos, utilizados na literatura para predições em vias livres, sob novas condições (vias urbanas com cruzamentos) e comparar a eficácia da predição de cada um deles.

Para alcançar o objetivo geral, temos os seguintes objetivos específicos foram definidos:

\begin{itemize}
    \item Implementar a análise e transformação dos dados para facilitar a aprendizagem;
    \item Implementar os modelos de aprendizagem de máquina escolhidos; 
    \item Realizar uma busca pelos melhores valores de parâmetros e hiper-parâmetros para cada modelo;
    \item Treinar os modelos com os dados processados;
    \item Automatizar a comparação dos modelos;
    \item Analisar o comportamento dos modelos quanto a distância no futuro da predição;
\end{itemize}

\section{Resultados Esperados} 

Ao fim deste trabalho, espera-se que os seguintes resultados tenham sido atingidos com sucesso:

\begin{itemize}
    \item  Os modelos implementados tenham melhor desempenho que os modelos de base de comparação;
    \item Os modelos utilizados apresentem pior desempenho à medida que as predições fiquem mais distantes no futuro;
    \item Os modelos utilizados tenham melhor desempenho quanto mais tempo do passado puderem receberem como entrada;
    \item Os modelos de aprendizagem profunda tenham um tempo de treinamento maior que os modelos tradicionais;
    \item Os modelos utilizados tenham um desempenho melhor quando recebem mais do que só o fluxo como entrada;
    \item Os modelos propostos sejam capazes de prever se o fluxo vai aumentar, ou diminuir de maneira mais precisa que os modelos de base.
    \item Os modelos utilizados são capazes de prever melhor que os modelos de base inferindo a partir de apenas algumas horas.
\end{itemize}

\section{Escopo}

Este trabalho tem como foco apenas dois cruzamentos de Brasília, mais especificamente, dois cruzamentos localizados na Avenida Hélio Prates. Além disso, o conjunto de dados se limita a apenas alguns meses. Dessa forma, os resultados podem ser distintos se os modelos forem utilizados em meses diferentes dos propostos no trabalho

Para o trabalho serão desconsiderados certos cenários em que a captura e o relacionamentos dos dados obtidos são muito imprecisos e/ou pontuais. Tais cenários podem influenciar nos resultados obtidos, causando uma disparidade significante entre a predição e a realidade. Para exemplificar, um possível cenário que será desconsiderado é a possibilidade de acontecer um acidente, ou uma obra nas rodovias, algo que pode modificar bastante o fluxo de veículos.

\section{Metodologia}

Será realizada a comparação dos modelos  \textit{\acrshort{LSTM}}, \textit{\acrshort{GRU}}, \textit{\acrshort{RF}} e \textit{\acrshort{SVM}}. Para tal, serão utilizados dados da infraestrutura urbana local, mais especificamente dados da fiscalização eletrônica coletada diretamente pelo \acrfull{DETRAN} do \acrfull{DF}. A comparação proposta consistirá de 4 etapas:

\begin{enumerate}
    \item Implementação dos Modelos: construção das arquiteturas;
    \item Pré-Processamento dos Dados: verificar, limpar e transformar os dados para serem utilizados de entrada nas arquiteturas propostas.
    \item Treinamento dos Modelos: implementação do espaço de treinamento dos modelos, considerando as propriedades de séries temporais;
    \item Escolha de Parâmetros: adaptação dos modelos para melhora dos resultados;
    \item Avaliação: análise das predições por meio do uso de métricas, para que seja possível comparar a eficácia dos diversos modelos.
\end{enumerate}

\section{Estrutura da Monografia}

Esta monografia está estruturado em 7 capítulos. Retirando o capítulo corrente de introdução, ainda temos:

\begin{itemize}
    \item \textbf{Capítulo 2 - Fundamentação Teórica:} teoria necessária para o entendimento do trabalho, com foco no funcionamento e nos conceitos por trás dos modelos escolhidos;
    \item \textbf{Capítulo 3 - Trabalhos Relacionados:} exposição e avaliação dos trabalhos similares e relacionados ao tema de previsão de fluxo;
    \item \textbf{Capítulo 4 - Análise de Dados:} exposição e análise do conjunto de registros de veículos, capturados por equipamentos de fiscalização eletrônica, utilizados no trabalho;
    \item \textbf{Capítulo 5 - Metodologia:} exposição detalhada da comparação dos modelos, das escolhas feitas nesse trabalho e de como os modelos reagem com diferentes valores de parâmetros;
    \item \textbf{Capítulo 6 - Resultados:} discussão dos resultados obtidos da comparação dos modelos;
    \item \textbf{Capítulo 7 - Conclusão:} conclusão do nosso trabalho e trabalhos futuros;
\end{itemize}
